\documentclass[a4paper, 10pt, ]{article}

\input{./COMMONFILES/preamble.tex}

% -----------------------------------------------------------------------------

\def\oznacenieCelku{Kolekcia učebných textov}

% -----------------------------------------------------------------------------


\def\KUTporadoveCislo{dev240822}

% \def\oznacenieVerzie{v0.9}
\def\oznacenieVerzie{\phantom{v1.0}}

\def\mesiacRok{august 2024}

\def\authorslabel{MT}






% -----------------------------------------------------------------------------

\begin{document}

% -----------------------------------------------------------------------------
% Uvodny nadpis

\noindent
\parbox[t][18mm][c]{0.3\textwidth}{%
\raisebox{-0.9\height}{%
\phantom{.}\includegraphics[height=18mm]{./COMMONFILES/URKFEIlogo.pdf}%
}%
}%
\parbox[t][18mm][c]{0.7\textwidth}{%
\raggedleft

\sffamily
\fontsize{16pt}{18pt}
\fontseries{sbc}
\selectfont

\noindent
\textcolor[rgb]{0.75, 0.75, 0.75}{\textls[25]{\oznacenieCelku}}
}%

\noindent
\parbox[t][16mm][b]{0.5\textwidth}{%
\raggedright

\color{Gray}
\sffamily

\fontsize{12pt}{12pt}
\selectfont
\mesiacRok

\fontsize{6pt}{10pt}
\selectfont
github.com/PracovnyBod/KUT

\fontsize{8pt}{10pt}
\selectfont
\authorslabel




}%
\parbox[t][16mm][b]{0.5\textwidth}{%
\raggedleft

\sffamily

\fontsize{6pt}{6pt}
\selectfont

\textcolor[rgb]{0.68, 0.68, 0.68}{\oznacenieVerzie}


\fontsize{14pt}{14pt}
\selectfont

\bfseries

\includegraphics[height=12pt]{./COMMONFILES/KUT_logo_v0.1.pdf}%
{%
\textls[-50]{\KUTporadoveCislo}
}%
}%

% -----------------------------------------------------------------------------




\vspace{6mm}

% ---------------------------------------------
\sffamily
\bfseries
\fontsize{18pt}{21pt}
\selectfont

\begin{flushleft}
    Systém prvého rádu:\\ vlastnosti a charakteristiky
\end{flushleft}

\bigskip

% -----------------------------------------------------------------------------
\normalsize
\normalfont
% -----------------------------------------------------------------------------












\noindent
\lettrine[lines=1, nindent=1pt, loversize=0.0]{C}{ieľom} 
textu je súhrn vlastností a charakteristík dynamického systému, ktorý má jeden vstupný signál $u(t)$ a jeden výstupný signál $y(t)$ a~tieto sú spojité v čase. Uvažuje sa lineárny, časovo invariantný dynamický systém.

Pojem \emph{rád systému} má v podstate rovnaký význam ako pri diferenciálnej rovnici. Diferenciálna rovnica $n$-tého rádu opisuje dynamický systém $n$-tého rádu. Dif. rovnica $n$-tého rádu je taká, v ktorej vystupuje maximálne $n$-tá derivácia neznámej. V kontexte prenosovej funkcie systému to znamená, že charakteristický polynóm systému je $n$-tého stupňa.

Osobitne uvedieme, že samozrejme uvažujeme \emph{kauzálny systém}, teda výstup systému je následkom diania v súčastnosti a minulosti. Z matematického hľadiska na prenosovú funkciu to znamená, že pre stupne polynómov $A(s)$ a $B(s)$ platí $n \geq m$ pričom charakteristický polynóm $A(s)$ má stupeň $n$, polynóm $B(s)$ má stupeň $m$ a uvažujme prenosovú funkciu v tvare
\begin{equation}
    G(s) = \frac{B(s)}{A(s)}
\end{equation}

Navyše, v praxi, pri matematickom modelovaní reálnych systémov, má v mnohých prípadoch význam hovoriť o systémoch, ktoré sami o sebe neobsahujú „zdroj energie“, sú len „energetickým spotrebičom“, sú \emph{energeticky disipatívne}. V takomto prípade pre prenosovú funkciu platí, že jej relatívny stupeň $n^\star = n-m$ je $n^\star \geq 1$.



\section{Matematický opis}


\subsection{Prenosová funkcia systému prvého rádu}

Vzhľadom na predpoklady uvedené vyššie, prenosová funkcia systému prvého rádu je vo všeobecnosti v tvare
\begin{equation}
    G(s) = \frac{b_0}{a_1 s + a_0}
\end{equation}
kde $b_0$, $a_1$ a $a_0$ reálne čísla. Typicky (a často veľmi užitočne) sa však uvádza $A(s)$ ako monický polynóm, taký, ktorý má pri najvyššej mocnine $s$ koeficient rovný $1$. Teda v~tomto prípade
\begin{equation}
    G(s) = \frac{b_0}{s + a_0}
\end{equation}
Pre úplnosť teda $B(s) = b_0$ je stupňa $m=0$ a $A(s) = s + a_0$ je stupňa $n=1$. Koeficienty týchto polynómov sú parametrami systému.



\subsection{Diferenciálna rovnica systému prvého rádu}

Aby sme nadviazali na predchádzajúci odsek a zároveň ukázali prepis systému z~prenosovej funkcie na diferenciálnu rovnicu, tak konštatujme, že
\begin{equation}
    G(s) = \frac{Y(s)}{U(s)}
\end{equation}
kde $Y(s)$ je Laplaceov obraz výstupného signálu a $U(s)$ je Laplaceov obraz vstupného signálu. V tomto prípade teda
\begin{subequations}
\begin{align}
    Y(s) &= G(s) U(s) = \frac{b_0}{s + a_0} U(s) \\
    \left(s + a_0\right) Y(s) &= b_0 U(s) \\
    s Y(s) + a_0 Y(s) &= b_0 U(s) \\
    s Y(s)  &= -  a_0 Y(s) b_0 U(s) 
\end{align}
\end{subequations}
a teda diferenciálna rovnica je
\begin{equation} \label{difrovnicanavsimnutie}
    \dot y(t) = - a_0 y(t) + b_0 u(t)
\end{equation}

Prepis opačným smerom, z dif. rovnice na prenosovú funkciu, je samozrejme štandardné aplikovanie Laplaceovej transformácie na rovnicu \eqref{difrovnicanavsimnutie} pri nulových začiatočných podmienkach. 





\subsection{Opis systému v stavovom priestore}

V stavovom priestore je potrebne zaviesť stavový vektor $x(t) \in \mathbb R^n$. Vo všeobecnosti je opis lineárneho systému v stavovom priestore v tvare
\begin{subequations}
\begin{align}
    \dot x(t) &= A x(t) + b u(t) \\
    y(t) &= c^\naT x(t) 
\end{align}
\end{subequations}
kde $A \in \mathbb R^{n \times n}$, $b \in \mathbb R^n$ a $c \in \mathbb R^n$ sú matica a vektory a ide o parametre systému. 

Pri stanovení vektora $x(t)$ ide vo všeobecnosti o prepis diferenciálnej rovnice vyššieho rádu na sústavu rovníc prvého rádu. V tomto prípade máme dif. rovnicu \eqref{difrovnicanavsimnutie} čo už je rovnica prvého rádu. Formálne teda zvoľme
\begin{equation}
    x_1(t) = y(t)
\end{equation}
a teda
\begin{equation}
    \dot x_1(t) = \dot y(t) = - a_0 x_1(t) + b_0 u(t)
\end{equation}
je vlastne „sústava“ jednej diferenciálnej rovnice. Formálne:
\begin{subequations}
\begin{align}
    \dot x_1(t) &= - a_0 x_1(t) + b_0 u(t) \\
    y(t) &= x_1(t)
\end{align}
\end{subequations}
je opis systému v stavovom priestore kde $x_1(t)$ je stavová veličina. Pre úplnosť, stavový vektor v tomto prípade je $x(t) = x_1(t)$ a matica $A = -a_0$, vektor $b = b_0$ a vektor $c = 1$.




\section{Stabilita}

TODO




















% -----------------------------------------------------------------------------

\end{document}

% -----------------------------------------------------------------------------