\documentclass[a4paper, 10pt, ]{article}

\input{./COMMONFILES/preamble.tex}

% -----------------------------------------------------------------------------

\def\oznacenieCelku{Kolekcia učebných textov}

% -----------------------------------------------------------------------------


\def\KUTporadoveCislo{dev240804}

% \def\oznacenieVerzie{v0.9}
\def\oznacenieVerzie{\phantom{v1.0}}

\def\mesiacRok{august 2024}

\def\authorslabel{MT}






% -----------------------------------------------------------------------------

\begin{document}

% -----------------------------------------------------------------------------
% Uvodny nadpis

\noindent
\parbox[t][18mm][c]{0.3\textwidth}{%
\raisebox{-0.9\height}{%
\phantom{.}\includegraphics[height=18mm]{./COMMONFILES/URKFEIlogo.pdf}%
}%
}%
\parbox[t][18mm][c]{0.7\textwidth}{%
\raggedleft

\sffamily
\fontsize{16pt}{18pt}
\fontseries{sbc}
\selectfont

\noindent
\textcolor[rgb]{0.75, 0.75, 0.75}{\textls[25]{\oznacenieCelku}}
}%

\noindent
\parbox[t][16mm][b]{0.5\textwidth}{%
\raggedright

\color{Gray}
\sffamily

\fontsize{12pt}{12pt}
\selectfont
\mesiacRok

\fontsize{6pt}{10pt}
\selectfont
github.com/PracovnyBod/KUT

\fontsize{8pt}{10pt}
\selectfont
\authorslabel




}%
\parbox[t][16mm][b]{0.5\textwidth}{%
\raggedleft

\sffamily

\fontsize{6pt}{6pt}
\selectfont

\textcolor[rgb]{0.68, 0.68, 0.68}{\oznacenieVerzie}


\fontsize{14pt}{14pt}
\selectfont

\bfseries

\includegraphics[height=12pt]{./COMMONFILES/KUT_logo_v0.1.pdf}%
{%
\textls[-50]{\KUTporadoveCislo}
}%
}%

% -----------------------------------------------------------------------------




\vspace{6mm}

% ---------------------------------------------
\sffamily
\bfseries
\fontsize{18pt}{21pt}
\selectfont

\begin{flushleft}
    O prenosovej funkcii
\end{flushleft}

\bigskip

% -----------------------------------------------------------------------------
\normalsize
\normalfont
% -----------------------------------------------------------------------------












\noindent
\lettrine[lines=1, nindent=1pt, loversize=0.0]{P}{renosová} 
funkcia je nástroj pre matematické modelovanie lineárnych časovo-invariantných dynamických systémov.

\section{Úvod}

Primárne sú dynamické systémy opisované diferenciálnymi rovnicami. Ak sú tieto rovnice lineárne, hovoríme, že systém, ktorý opisujú, je lineárny. Ak koeficienty v dif. rovnici nie sú funkciami času, hovoríme, že systém je časovo invariantný.

Vo všeobecnosti hľadáme riešenie dif. rovnice. V kontexte dynamických systémov je riešením dif. rovnice funkcia času. Z hľadiska systému hovoríme, že táto funkcia je výstupným signálom systému. Na hľadané riešenie má vplyv niekoľko faktorov. Vo všeobecnosti je riešenie dané samozrejme samotnou dif. rovnicou, jej rádom a~hodnotami jej koeficientov. Konkrétne riešenia sú potom dané začiatočnými podmienkami a~vstupným signálom systému.

Z hľadiska systému hovoríme o ráde dif. rovnice a~o jej koeficientoch ako o~parametroch systému. Hovoriť o začiatočných podmienkach systému má samozrejme tiež význam. Napokon nás zaujíma vplyv vstupného signálu na výstupný signál systému a~s matematickým modelovaním tohto vplyvu súvisí pojem prenosová funkcia. Obrazne hovoríme o prenose zo vstupu na výstup systému.



\section{Náčrt analytického prístupu k pojmu prenosová funkcia}

Obdobne ako pri hľadaní riešenia homogénnej dif. rovnice, kde sa ko východisko predpokladá riešenie v tvare exponenciálnej funkcie $e^{s t}$, tak pri hľadaní riešenia nehomogénnej dif. rovnice je možné skúmať predpoklad, že vstupný signál je v tvare exponenciálnej funkcie $e^{s t}$. 

Najskôr pripomeňme, že riešením homogénnej dif. rovnice 
\begin{equation}
    \dot y(t) + a y(t) = 0 \qquad y(0) = y_0
\end{equation}
je
\begin{equation}
    y(t) =  e^{-a t} y_0
\end{equation}
a~ide tu o rovnicu prvého rádu. 

Formálne je však aj tu možné uplatniť rozklad dif. rovnice vyššieho rádu na sústavu rovníc prvého rádu v zmysle
\begin{subequations}
    \begin{align}
        \dot x(t) &= a x(t) \qquad x(0) = x_0 \\
        y(t) &= x(t)
    \end{align}    
\end{subequations}
kde $a \in \mathbb{R}$ a $x(t)$ je stavová veličina. Pri dif. rovnici vyššieho rádu by $x(t)$ bol vektor stavových veličín a~udával by sústavu rovníc v tvare
\begin{subequations}
    \begin{align}
        \dot x(t) &= A x(t) \qquad x(0) = x_0  \\
        y(t) &= c^\naT x(t)
    \end{align}    
\end{subequations}
kde $A \in \mathbb{R}^{n\times n} $ je matica, $c \in \mathbb{R}^{n}$ je vektor a~$x_0 \in \mathbb{R}^{n}$ je vektor. Riešením je
\begin{equation}
    y(t) = c^\naT e^{A t} x_0
\end{equation}
kde sme využili objekt $e^{At}$ čo je tzv. maticová exponenciálna funkcia. Tu sa jej definícii nebudeme venovať podrobne, čitateľa odkazujeme napr. na \cite{Aastroem2020}. Ide zjavne o zovšeobecnenie skalárneho prípadu (systémy prvého rádu) pre vektorový prípad (systémy vyššieho rádu). Definícia a~následné využívanie matice $e^{At}$ je základom pre pojmy ako fundamentálne riešenia systému (diferenciálnej rovnice). Samotná matica $e^{At}$ sa označuje napríklad aj ako matica fundamentálnych riešení. Takpovediac „účinok“ matice $e^{At}$ je daný maticou $A$, a~tú možno charakterizovať jej vlastnými číslami (a~vlastnými vektormi). Tieto sú následne zdrojom definície pojmu charakteristická rovnica tak ako sa to využíva pri hľadaní analytického riešenia diferenciálnej rovnice.

V prípade nehomogénnej dif. rovnice je systém daný sústavou rovníc v tvare
\begin{subequations} \label{sysLTIstavpries}
    \begin{align}
        \dot x(t) &= A x(t) + b u(t) \qquad x(0) = x_0 \\
        y(t) &= c^\naT x(t)
    \end{align}
\end{subequations}
kde $u(t)$ je vstupný signál, $b \in \mathbb{R}^{n}$ je vektor. Je možné ukázať, že
\begin{align} \label{xriesLTI}
	x(t) = e^{At} x(0) + \int_0^t e^{A(t-\tau)} b u(\tau) \text{d}\tau
\end{align}
a teda samotné riešenie (výstupný signál $y(t)$) je
\begin{subequations}
    \begin{align}
        y(t) &= c^\naT x(t) \\
        y(t) &= c^\naT e^{At} x(0) + \int_0^t c^\naT e^{A(t-\tau)} b u(\tau) \text{d}\tau \label{riesnhrov}
    \end{align}
\end{subequations}
Prvý člen (na pravej strane rovnice \eqref{riesnhrov}) sa nazýva \emph{vlastná zložka riešenia} (je vyvolaná začiatočnými podmienkami) a druhý člen sa nazýva  \emph{vnútená zložka riešenia} (je vyvolaná vstupným signálom).

Ako sme uviedli, zámerom je skúmať predpoklad, že vstupný signál je v tvare exponenciálnej funkcie
\begin{align}
	u(t) = e^{st}
\end{align}
kde $s = \sigma + j\omega$ (vo všeobecnosti). To, že $s$ je komplexné číslo (komplexná premenná) umožňuje považovať tento špeciálny signál vlastne za triedu signálov (rôzneho typu). Reálna časť premennej $s$ určuje exponenciálny rast alebo pokles (dokonca ak $s = 0$ potom je špeciálny signál vlastne konštantným) a imaginárna časť určuje harmonické kmitanie signálu.

Máme \eqref{xriesLTI}, a teda:
\begin{align}
	x(t) = e^{At} x(0) + \int_0^t \left( e^{A(t-\tau)} b e^{s\tau} \right) \text{d}\tau
\end{align}
kde pri manipulácii s výrazom $ \left( e^{A(t-\tau)} b e^{s\tau} \right)$ treba manipulovať s ohľadom na fakt, že ide o matice a vektory. V každom prípade, po integrácii sa získa
\begin{align}
	x(t) = e^{At} x(0) + e^{At} \left( sI - A \right)^{-1}  \left( e^{(sI-A)t} - I \right) b
\end{align}
kde $I$ je jednotková matica.

Celkové riešenie, inými slovami výstupný signál systému, potom je
\begin{align}
	\begin{aligned}
		y(t) &= c^\naT e^{At} x(0) + c^\naT e^{At} \left( sI - A \right)^{-1}  \left( e^{(sI-A)t} - I \right) b \\
		&= c^\naT e^{At} x(0) + c^\naT e^{At} \left( sI - A \right)^{-1}  \left( e^{st} e^{-At} - I \right) b \\
		&= c^\naT e^{At} x(0) + c^\naT e^{At} \left( sI - A \right)^{-1}  \left( e^{st} e^{-At}b - b \right) \\
		&= c^\naT e^{At} x(0) +   \left(c^\naT e^{At} \left( sI - A \right)^{-1} e^{st} e^{-At}b - c^\naT e^{At} \left( sI - A \right)^{-1} b \right) \\
		&= c^\naT e^{At} x(0) +   \left(c^\naT \left( sI - A \right)^{-1} e^{st} b - c^\naT e^{At} \left( sI - A \right)^{-1} b \right)
	\end{aligned}
\end{align}
V tomto bode je možné konštatovať:
\begin{align}
	\begin{aligned}
		y(t)
		&=
		\underbrace{
		c^\naT e^{At} x(0)
		}_{\text{vlastná zložka}}
		+
		\underbrace{
		\left(c^\naT \left( sI - A \right)^{-1} e^{st} b - c^\naT e^{At} \left( sI - A \right)^{-1} b \right)
		}_{\text{vnútená zložka}}
	\end{aligned}
\end{align}
a zároveň:
\begin{align}
    \begin{split}
        y(t) 
        &= 
        c^\naT e^{At} \left( x(0) -  \left( sI - A \right)^{-1} b\right) +   \left(c^\naT  \left( sI - A \right)^{-1}  b e^{st} \right) 
        \\
        &=
        \underbrace{
        c^\naT e^{At} \left( x(0) -  \left( sI - A \right)^{-1} b\right)
        }_{\text{zložka opisujúca prechodné deje}}
        +
        \underbrace{
        \left(c^\naT  \left( sI - A \right)^{-1}  b  \right) e^{st}
        }_{\text{čisto exponenciálna zložka}}
    \end{split}
\end{align}

O vplyve samotného špeciálneho signálu $e^{st}$ na celkové riešenie teda rozhoduje výraz $c^\naT  \left( sI - A \right)^{-1}  b$. Formálne sa
\begin{equation}
	G(s) = c^\naT  \left( sI - A \right)^{-1}  b
\end{equation}
nazýva prenosová funkcia systému.

Uvedené je založené na fakte vyjadrenom všeobecným riešením \eqref{xriesLTI} pričom ide o~riešenie sústavy dif. rovníc prvého rádu v tvare \eqref{sysLTIstavpries}. Takpovediac pôvodná dif. rovnica vyššieho rádu je pre tento prípad v tvare
\begin{equation} \label{vseobDifRov_nh2}
	\frac{\text{d}^n y(t)}{\text{d}t^n} + a_{n-1} \frac{\text{d}^{(n-1)} y(t)}{\text{d}t^{(n-1)}} + \cdots + a_0 y(t) = b_m \frac{\text{d}^m u(t)}{\text{d}t^m} + b_{m-1} \frac{\text{d}^{m-1} u(t)}{\text{d}t^{m-1}} + \cdots + b_0 u(t)
\end{equation}
Potom ak na vstupe uvažujeme $u(t) = e^{st}$ a zároveň vieme, že riešenie systému je tiež nejaký exponenciálny signál, čo možno vo všeobecnosti vyjadriť ako $y(t) = y_0 e^{st}$ (kde $y_0$ najmä odlišuje $y(t)$ od $u(t)$). Ak $y(t)$ a $u(t)$ dosadíme do \eqref{vseobDifRov_nh2}, vidíme, že
\begin{equation}
    \begin{aligned}
                \frac{\text{d}^n y_0 e^{st}}{\text{d}t^n}
        + a_{n-1} \frac{\text{d}^{(n-1)} y_0 e^{st}}{\text{d}t^{(n-1)}}
        + \cdots
        + a_0 y_0 e^{st}
        &=
        b_m \frac{\text{d}^m e^{st}}{\text{d}t^m}
        + b_{m-1} \frac{\text{d}^{m-1} e^{st}}{\text{d}t^{m-1}}
        + \cdots
        + b_0 e^{st}
        \\
        y_0 e^{st} s^n
        + a_{n-1}  y_0 e^{st} s^{(n-1)}
        + \cdots
        + a_0 y_0 e^{st}
        &=
        b_m  e^{st} s^m
        + b_{m-1}  e^{st} s^{m-1}
        + \cdots
        + b_0 e^{st}
        \\
        \left(
            s^n
            + a_{n-1}   s^{(n-1)}
            + \cdots
            + a_0
        \right)
        y_0 e^{st}
        &=
        \left(
        b_m   s^m
        + b_{m-1}   s^{m-1}
        + \cdots
        + b_0
        \right)
        e^{st}
        \\
        y_0 e^{st}
        &=
        \frac{
        \left(
        b_m   s^m
        + b_{m-1}   s^{m-1}
        + \cdots
        + b_0
        \right)
        }{
        \left(
            s^n
            + a_{n-1}   s^{(n-1)}
            + \cdots
            + a_0
        \right)
        }
        e^{st}
    \end{aligned}
\end{equation}
a teda môžeme povedať, že riešenie systému závislé od špeciálneho signálu $e^{st}$ je
\begin{equation}
    y(t)
    =
    \frac{
    \left(
    b_m   s^m
    + b_{m-1}   s^{m-1}
    + \cdots
    + b_0
    \right)
    }{
    \left(
    s^n
    + a_{n-1}   s^{(n-1)}
    + \cdots
    + a_0
    \right)
    }
    e^{st}
\end{equation}

Označme
\begin{subequations}
	\begin{align}
		B(s) &= \left( b_m   s^m + b_{m-1}   s^{m-1} + \cdots + b_0 \right) \\
		A(s) &=  \left( s^n + a_{n-1}   s^{(n-1)} + \cdots + a_0 \right)
	\end{align}
\end{subequations}
a výraz
\begin{align}
	G(s) = \frac{B(s)}{A(s)}
\end{align}
vyjadruje prenosovú funkciu systému.



\section{Definícia s využitím Laplaceovej transformácie}



















\printbibliography[title={Referencie a ďalšia literatúra}]
% -----------------------------------------------------------------------------

\end{document}

% -----------------------------------------------------------------------------