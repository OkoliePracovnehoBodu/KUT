\documentclass[a4paper, 10pt, ]{article}

\input{./COMMONFILES/preamble.tex}

% -----------------------------------------------------------------------------

\def\oznacenieCelku{Kolekcia učebných textov}

% -----------------------------------------------------------------------------


\def\KUTporadoveCislo{dev240808}

% \def\oznacenieVerzie{v0.9}
\def\oznacenieVerzie{\phantom{v1.0}}

\def\mesiacRok{august 2024}

\def\authorslabel{MT}






% -----------------------------------------------------------------------------

\begin{document}

% -----------------------------------------------------------------------------
% Uvodny nadpis

\noindent
\parbox[t][18mm][c]{0.3\textwidth}{%
\raisebox{-0.9\height}{%
\phantom{.}\includegraphics[height=18mm]{./COMMONFILES/URKFEIlogo.pdf}%
}%
}%
\parbox[t][18mm][c]{0.7\textwidth}{%
\raggedleft

\sffamily
\fontsize{16pt}{18pt}
\fontseries{sbc}
\selectfont

\noindent
\textcolor[rgb]{0.75, 0.75, 0.75}{\textls[25]{\oznacenieCelku}}
}%

\noindent
\parbox[t][16mm][b]{0.5\textwidth}{%
\raggedright

\color{Gray}
\sffamily

\fontsize{12pt}{12pt}
\selectfont
\mesiacRok

\fontsize{6pt}{10pt}
\selectfont
github.com/PracovnyBod/KUT

\fontsize{8pt}{10pt}
\selectfont
\authorslabel




}%
\parbox[t][16mm][b]{0.5\textwidth}{%
\raggedleft

\sffamily

\fontsize{6pt}{6pt}
\selectfont

\textcolor[rgb]{0.68, 0.68, 0.68}{\oznacenieVerzie}


\fontsize{14pt}{14pt}
\selectfont

\bfseries

\includegraphics[height=12pt]{./COMMONFILES/KUT_logo_v0.1.pdf}%
{%
\textls[-50]{\KUTporadoveCislo}
}%
}%

% -----------------------------------------------------------------------------




\vspace{6mm}

% ---------------------------------------------
\sffamily
\bfseries
\fontsize{18pt}{21pt}
\selectfont

\begin{flushleft}
    Algebra prenosových funkcií
\end{flushleft}

\bigskip

% -----------------------------------------------------------------------------
\normalsize
\normalfont
% -----------------------------------------------------------------------------












\noindent
\lettrine[lines=1, nindent=1pt, loversize=0.0]{P}{renosová} 
funkcia je nástroj pre matematické modelovanie lineárnych časovo-invariantných dynamických systémov. Prenosovú funkciu je možné vidieť aj ako jeden blok v blokovej schéme, teda:


\begin{center}

    \makebox[\textwidth][c]{%
    \input{../SVG/TFalgebra_lenG.pdf_tex}
    }

	\figcaption{Prenosová funkcia ako jeden blok v blokovej schéme}
	\label{TFalgebra_lenG}

\end{center}

Manipulácia s takýmito blokmi je jednou z aplikácií algebry prenosových funkcií. V tomto zmysle je potrebné uvažovať tri základné situácie. Sériové zapojenie blokov, paralelné zapojenie blokov a spätnoväzbové zapojenie blokov.



\paragraph{Sériové zapojenie blokov}

Uvažujme systém, ktorý je tvorený kaskádnou kombináciou dvoch podsystémov. Prenosové funkcie podsystémov sú $G_1(s)$ a $G_2(s)$. Vstup prvého podsystému je zároveň vstupom celkového systému. Výstup prvého podsystému je vstupom druhého podsystému. Výstup druhého podsystému je zároveň výstupom celkového systému. Ide o sériové zapojenie podsystémov.

\begin{center}

    \makebox[\textwidth][c]{%
    \input{../SVG/TFalgebra_seriove.pdf_tex}
    }

	\figcaption{Sériové zapojenie blokov}
	\label{TFalgebra_seriove}

\end{center}

Hľadáme prenosovú funkciu celkového systému, označme ju $G(s)$. Pre sériové zapojenie podsystémov platí
\begin{align}
    G(s) = G_1(s)\ G_2(s)
\end{align}
Výslednú prenosovú funkciu teda získame súčinom prenosových funkcií podsystémov.



\paragraph{Paralelné zapojenie blokov}

\begin{center}

    \makebox[\textwidth][c]{%
    \input{../SVG/TFalgebra_paralelne.pdf_tex}
    }

	\figcaption{Paralelné zapojenie blokov}
	\label{TFalgebra_paralelne}

\end{center}

Pri paralelnom zapojení podsystémov s prenosovými funkciami $G_1(s)$ a $G_2(s)$ je výstupom celkového systému jednoducho súčet výstupov podsystémov. Pre prenosovú funkciu celkového systému $G(s)$ platí
\begin{align}
    G(s) = G_1(s) + G_2(s)
\end{align}



\paragraph{Spätnoväzbové zapojenie blokov}

Spätnoväzbové zapojenie blokov je znázornené na obr.~\ref{TFalgebra_spatnovazbove}. Pre lepšiu orientáciu je vstup celkového systému označený ako $u$ a výstup celkového systému ako $y$. Signál $y$ je vstupom spätnoväzbového podsystému $G_2(s)$. Takáto spätná väzba je odčítavaná (ide o zápornú spätnú väzbu) od vstupného signálu $u$. Vzniká odchýlkový signál $e$, ktorý je vstupom podsystému $G_1(s)$.


\begin{center}

    \makebox[\textwidth][c]{%
    \input{../SVG/TFalgebra_spatnovazbove.pdf_tex}
    }

	\figcaption{Spätnoväzbové zapojenie blokov}
	\label{TFalgebra_spatnovazbove}

\end{center}

Bez uvádzania podrobností a predpokladov môžeme písať o odchýlkovom signále:
\begin{align}
    e = u - G_2(s) y
\end{align}
a potom
\begin{subequations}
    \begin{align}
        y &= G_1(s) e \\
        y &= G_1(s) \left( u - G_2(s) y \right) \\
        \left( 1 + G_1(s)G_2(s) \right) y &= G_1(s) u  \\
        y &= \frac{G_1(s)}{\left( 1 + G_1(s)G_2(s) \right)} u
    \end{align}
\end{subequations}
Pre prenosovú funkciu celkového systému $G(s)$ platí
\begin{align}
   G(s) = \frac{G_1(s)}{\left( 1 + G_1(s)G_2(s) \right)}
\end{align}





% -----------------------------------------------------------------------------

\end{document}

% -----------------------------------------------------------------------------