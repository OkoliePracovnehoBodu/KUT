\documentclass[a4paper, 10pt, ]{article}

\input{./COMMONFILES/preamble.tex}

% -----------------------------------------------------------------------------

\def\oznacenieCelku{Kolekcia učebných textov}

% -----------------------------------------------------------------------------


\def\KUTporadoveCislo{006}

% \def\oznacenieVerzie{v0.9}
\def\oznacenieVerzie{\phantom{v1.0}}

\def\mesiacRok{júl 2024}

\def\authorslabel{MT}







% -----------------------------------------------------------------------------

\begin{document}

% -----------------------------------------------------------------------------
% Uvodny nadpis

\noindent
\parbox[t][18mm][c]{0.3\textwidth}{%
\raisebox{-0.9\height}{%
\phantom{.}\includegraphics[height=18mm]{./COMMONFILES/URKFEIlogo.pdf}%
}%
}%
\parbox[t][18mm][c]{0.7\textwidth}{%
\raggedleft

\sffamily
\fontsize{16pt}{18pt}
\fontseries{sbc}
\selectfont

\noindent
\textcolor[rgb]{0.75, 0.75, 0.75}{\textls[25]{\oznacenieCelku}}
}%

\noindent
\parbox[t][16mm][b]{0.5\textwidth}{%
\raggedright

\color{Gray}
\sffamily

\fontsize{12pt}{12pt}
\selectfont
\mesiacRok

\fontsize{6pt}{10pt}
\selectfont
github.com/PracovnyBod/KUT

\fontsize{8pt}{10pt}
\selectfont
\authorslabel




}%
\parbox[t][16mm][b]{0.5\textwidth}{%
\raggedleft

\sffamily

\fontsize{6pt}{6pt}
\selectfont

\textcolor[rgb]{0.68, 0.68, 0.68}{\oznacenieVerzie}


\fontsize{14pt}{14pt}
\selectfont

\bfseries

\includegraphics[height=12pt]{./COMMONFILES/KUT_logo_v0.1.pdf}%
{%
\textls[-50]{\KUTporadoveCislo}
}%
}%

% -----------------------------------------------------------------------------




\vspace{6mm}

% ---------------------------------------------
\sffamily
\bfseries
\fontsize{18pt}{21pt}
\selectfont

\begin{flushleft}
	O riešení homogénnej lineárnej diferenciálnej rovnice druhého rádu s konštantnými koeficientami
\end{flushleft}

\bigskip

% -----------------------------------------------------------------------------
\normalsize
\normalfont
% -----------------------------------------------------------------------------












\noindent
\lettrine[lines=1, nindent=1pt, loversize=0.0]{T}{ento} 
text je spracovaný najmä podľa učebného materiálu \cite{Mihalikova2012}. Cieľom je poskytnúť prehľad v oblasti klasického spôsobu riešenia homogénnej lineárnej diferenciálnej rovnice druhého rádu s konštantnými koeficientami.

V princípe je možné tu uvedenú metódu aplikovať aj na diferenciálne rovnice iného rádu ako druhého. Rovnica druhého rádu však vhodne ilustruje podstatu hľadania riešenia.


\section{Štruktúra riešení homogénnej diferenciálnej rovnice}

Najprv zdôrazníme prívlastok \emph{homogénna} v zmysle, že predmetom tohto textu je \emph{všeobecné riešenie} dif. rovnice. Takéto riešenie je možné ďalej konkretizovať v zmysle začiatočných (prípadne okrajových) podmienok. Nezaoberáme sa tu tzv. \emph{partikulárnym} riešením, ktoré je dané prítomnosťou inej funkcie (času) ako je hľadaná neznáma funkcia.

Uvažujme diferenciálnu rovnicu v tvare
\begin{equation}
    \ddot y(t) + a_1 \dot y(t) + a_0 y(t) = 0 \label{e03}
\end{equation}
kde $y(t)$ je neznáma funkcia času. Túto funkciu hľadáme tak aby bola riešením rovnice \eqref{e03}, teda tak aby po dosadení $y(t)$ a jej derivácií do rovnice \eqref{e03} bola rovnica platná. Pre úplnosť, $a_1 \in \mathbb R$ a $a_2  \in \mathbb R$ sú konštantné koeficienty.

\subsection{Fundamentálne riešenia}

Všeobecné riešenie dif. rovnice \eqref{e03} nájdeme tak, že najprv hľadáme dve lineárne nezávislé riešenia. Tieto riešenia nazývame \emph{fundamentálnymi riešeniami}. Ich lineárna kombinácia je potom všeobecným riešením dif. rovnice \eqref{e03}.

Nech $y_1(t)$ a $y_2(t)$ sú dve riešenia dif. rovnice \eqref{e03}. Potom každá ich lineárna kombinácia $c_1 y_1(t) + c_2 y_2(t) $, $c_1, c_2 \in \mathbb R$ je tiež riešením dif. rovnice \eqref{e03} (vyplýva to z~lineárnosti derivácie). 

Špeciálne, ak $c_1 = c_2 = 0$ dostaneme riešenie $y(t) = 0$. Nazýva sa nulovým riešením alebo \emph{triviálnym riešením}.

Urobme nasledujúcu úvahu. Zoberme dve riešenia $y_1(t)$ a $y_2(t)$, ktoré spĺňajú začiatočné podmienky
\begin{subequations} \label{eIC}
    \begin{align}
        y_1(0) &= 1      &        y_2(0) &= 0       \\
        \dot y_1(0) &= 0  &      \dot y_2(0) &= 1
    \end{align}
\end{subequations}
Tieto riešenia sú tým určené jednoznačne v zmysle, že žiadne z nich nie je násobkom druhého. Tieto dve riešenia sú teda lineárne nezávislé.

Majme teda dve riešenia $y_1(t)$ a $y_2(t)$, a uvažujme ďalšie riešenie $y_3(t)$ spĺňajúce začiatočné podmienky
\begin{subequations}
    \begin{align}
        y_3(0) &= y_0           \\
        \dot y_3(0) &= z_0  
    \end{align}
\end{subequations}
Skúsme ukázať, že riešenia $y_1(t)$,  $y_2(t)$ a $y_3(t)$ sú navzájom lineárne závislé, teda existujú $\alpha, \beta \in \mathbb R$ také, že
\begin{equation}
    y_3(t) = \alpha y_1(t) + \beta y_2(t) \label{eY3}
\end{equation}
Dosadením začiatočných podmienok do rovnice \eqref{eY3} dostaneme
\begin{subequations}
    \begin{align}
        y_0 &= \alpha y_1(0) + \beta y_2(0)    \\
        z_0 &= \alpha \dot y_1(0) + \beta \dot y_2(0) 
    \end{align}
\end{subequations}
Čísla $\alpha$ a $\beta$ sú riešením tejto sústavy rovníc a sústava má jediné riešenie ak determinant tejto sústavy je nenulový.
\begin{equation} \label{Wdet}
    \begin{vmatrix}
        y_1(0) & y_2(0) \\
        \dot y_1(0) & \dot y_2(0)
    \end{vmatrix} 
    =
    y_1(0) \dot y_2(0) -  y_2(0) \dot y_1(0)
    \neq 0
\end{equation}
Pre prípad \eqref{eIC} je determinant rovný $1$.

Ak by sme predpokladali, že 
\begin{subequations} \label{eIC2}
    \begin{align}
        y_1(0) &= 1      &        y_2(0) &\neq 0       \\
        \dot y_1(0) &\neq 0  &      \dot y_2(0) &= 1
    \end{align}
\end{subequations}
tak determinant \eqref{Wdet} môže byť nulový. V tejto situácii ak $y_2(0) = 0$ a $\dot y_2(0) = 0$ tak $y_2(t) = 0$. Taktiež ak $y_2(0) = 0$ a $\dot y_2(0) \neq 0$ a $y_1(0) = 0$ tak existuje $c \in \mathbb R$ také, že
\begin{align}
    y_1(0) &= c y_2(0) \\
    \dot y_1(0) &= c \dot y_2(0)
\end{align}
a riešenia $y_1(t)$ a $y_2(t)$ by boli lineárne závislé. Pripomeňme, že hovoríme o situácii, že determinant \eqref{Wdet} je nulový. 

Inými slovami, pri nulovom determinante \eqref{Wdet} môžeme písať
\begin{equation}
    \frac{y_1(0)}{y_2(0)} = \frac{\dot y_1(0)}{\dot y_2(0)} = c
\end{equation}
Zároveň vieme, že $y_2(t)$ je riešenie dif. rovnice. Potom aj $\tilde y(t) = c y_2(t)$ je riešenie a~tiež platí $\tilde y(0) = c y_2(0)$, $\dot{\tilde{y}}(0) = c \dot y_2(0)$ a teda je možné aby $\tilde y(t) = y_1(t)$. Znamenalo by to, že $y_1(t)$ a $y_2(t)$ sú lineárne závislé, čo je v rozpore s východiskom, že sú lineárne nezávislé.

Ukázali sme, že determinant \eqref{Wdet} je nenulový, tak riešenia $y_1(t)$ a $y_2(t)$ sú lineárne nezávislé.


Ďalej je možné konštatovať, že mať jedno riešenie v tomto prípade nestačí. Potrebujeme aspoň dve riešenia. Všeobecné riešenie, alebo inak povedané, všetky riešenia dif. rovnice \eqref{e03}, získame ako lineárnu kombináciu dvoch lineárne nezávislých riešení.

Dve lineárne nezávislé riešenia dif. rovnice nazývame \emph{fundamentálnymi riešeniami} alebo \emph{bázou riešení} dif. rovnice (pozri tiež \cite{Kuben1995}). Je možné ukázať, že riešenia dif. rovnice tvoria vektorový priestor a ten má vždy bázu \cite{Kuben1995}.

Determinant \eqref{Wdet} sa nazýva Wronského determinant, v tomto prípade funkcií $y_1(t)$ a $y_2(t)$, a rozhoduje o lineárnej nezávislosti týchto funkcií.

Všeobecným riešením $y(t)$ nazývame lineárnu kombináciu dvoch fundamentálnych riešení $y_{f1}(t)$ a $y_{f2}(t)$, teda $y(t) = c_1 y_{f1}(t) + c_2 y_{f2}(t)$, kde $c_1, c_2 \in \mathbb R$ sú ľubovoľné konštanty.

Nepoznáme všeobecnú metódu, pomocou ktorej by sme vždy vedeli nájsť fundamentálne riešenia. V praxi sa využívajú rôzne postupy, ktoré sú zvyčajne založené na skúšaní rôznych tvarov riešení. Pri lineárnej diferenciálnej rovnici s konštantnými koeficientmi je možné nájsť riešenia v tvare exponenciálnej funkcie. V prípade dif. rovnice \eqref{e03} je možné hľadať riešenie v tvare $y(t) = e^{s t}$, kde $s \in \mathbb C$.



\section{Všeobecné riešenie homogénnej diferenciálnej rovnice}

Uvažujme diferenciálnu rovnicu v tvare
\begin{equation}
    \ddot y(t) + a_1 \dot y(t) + a_0 y(t) = 0 \label{e01}
\end{equation}
kde $a_1 \in \mathbb R$ a $a_2  \in \mathbb R$.

Riešenie rovnice \eqref{e01} hľadajme v tvare 
\begin{equation}
    y(t) = e^{s t}
\end{equation}
kde $s \in \mathbb C$. Potom
\begin{align}
    \dot y(t) &= s e^{s t} \\
    \ddot y(t) &= s^2 e^{s t}
\end{align}
Dosadením do rovnice \eqref{e01} máme
\begin{subequations}
    \begin{align}
        s^2 e^{s t} + a_1 s e^{s t} + a_0 e^{s t} &= 0 \\
        e^{s t} (s^2 + a_1 s + a_0) &= 0
    \end{align}
\end{subequations}
Keďže $e^{s t} \neq 0$, pretože nehľadáme triviálne riešenie, tak pre nájdenie riešenia musí platiť
\begin{equation}
    s^2 + a_1 s + a_0 = 0 \label{e02}
\end{equation}


\paragraph{Charakteristický polynóm a korene charakteristického polynómu}

Rovnica \eqref{e02} sa nazýva \emph{charakteristická rovnica} dif. rovnice \eqref{e01} a~jej riešenia $s_1$ a~$s_2$ sú \emph{charakteristickými číslami} dif. rovnice \eqref{e01}. Inými slovami polynóm $s^2 + a_1 s + a_0$ sa nazýva \emph{charakteristický polynóm} dif. rovnice \eqref{e01}.

Funkcia $e^{s t}$ je riešením dif. rovnice \eqref{e01} ak $s$ je riešením algebraickej rovnice \eqref{e02}. Inými slovami, $e^{s t}$ je riešením dif. rovnice \eqref{e01} ak $s$ je koreňom charakteristického polynómu dif. rovnice. 


Vo všeobecnosti môžeme určiť toľko navzájom odlišných funkcií $e^{s t}$, teda riešení diferenciálnej rovnice, koľko koreňov má charakteristický polynóm. V tomto prípade dif. rovnice druhého rádu sú to dva korene. Môžu byť
\begin{itemize}
    \item dva navzájom rôzne reálne korene,
    \item jeden dvojnásobný reálny koreň,
    \item dva komplexne združené korene.
\end{itemize}
Tieto prípady vedú k rôznym typom riešení dif. rovnice.








\subsection{Prípad: dva navzájom rôzne reálne korene}


Nech charakteristický polynóm má dva navzájom rôzne reálne korene $s_1$ a $s_2$. Potom všeobecné riešenie dif. rovnice \eqref{e01} je
\begin{equation}
    y(t) = c_1 e^{s_1 t} + c_2 e^{s_2 t}
\end{equation}
kde uvažujeme, že fundamentálnymi riešeniami sú
\begin{subequations} \label{kfunr}
    \begin{align}
        y_{f1}(t) &= e^{s_1 t} \\
        y_{f2}(t) &= e^{s_2 t}
    \end{align} 
\end{subequations}
Ich Wronskián (Wronského determinant) je
\begin{equation}
    \begin{aligned}
        W(t) &= \begin{vmatrix}
            e^{s_1 t} & e^{s_2 t} \\
            s_1 e^{s_1 t} & s_2 e^{s_2 t}
        \end{vmatrix} \\
        &=e^{s_1t} s_2 e^{s_2t} - e^{s_2} s_1 e^{s_1t} \\
        &= e^{(s_1 + s_2) t} (s_2 - s_1) \neq 0
    \end{aligned}
\end{equation}
čím sme ukázali, že sú lineárne nezávislé a potvrdili, že všeobecné riešenie dif. rovnice \eqref{e01} v tomto prípade je
\begin{equation}
    y(t) = c_1 e^{s_1 t} + c_2 e^{s_2 t}
\end{equation}

\subsection{Prípad: jeden dvojnásobný reálny koreň}

Nech charakteristický polynóm má koreň $s$. Riešením dif. rovnice \eqref{e01} je $y_{f1} = e^{st}$. Pre nájdenie všeobecného riešenia však potrebujeme aj druhé fundamentálne riešenie. Ukážme, že $y_{f2} = t e^{st}$ je tiež riešením dif. rovnice \eqref{e01}. Platí
\begin{subequations}
    \begin{align}
        \dot y_{f2} &= e^{st} + s t e^{st} \\
        \begin{split}
            \ddot y_{f2} &=  e^{st} s + s \left( e^{st} +  t e^{st} s \right) \\
            &= s e^{st} + s e^{st} + s^2 t e^{st} \\
            &= 2 s e^{st} + s^2 t e^{st}
        \end{split}
    \end{align}
\end{subequations}
Dosadením do dif. rovnice \eqref{e01}
\begin{subequations}
    \begin{align}
        \left( 2 s e^{st} + s^2 t e^{st} \right) + a_1 \left( e^{st} + s t e^{st} \right) + a_0 t e^{st} &= 0 \\
        2 s e^{st} + s^2 t e^{st} + a_1 e^{st} + a_1 s t e^{st} + a_0 t e^{st} &= 0 \\
        e^{st} \left( 2 s + s^2 t + a_1 + a_1 s t + a_0 t \right) &= 0 \\
        e^{st} \left(  t \left( s^2 + a_1s + a_0 \right) + 2s + a_1   \right) &= 0
    \end{align}
\end{subequations}
Keďže
\begin{align}
    D &= a_1^2 - 4 a_0 = 0 \\
    s &= -\frac{a_1}{2}
\end{align}
tak
\begin{align}
    2s+ a_1 = 2 \left( -\frac{a_1}{2} \right) + a_1 = 0
\end{align}
a teda $y_{f2} = t e^{st}$ je riešením dif. rovnice \eqref{e01}.
Máme teda dve fundamentálne riešenia v tvare
\begin{subequations}
    \begin{align}
        y_{f1} &= e^{st} \\
        y_{f2} &= t e^{st}
    \end{align}
\end{subequations}
a ostáva ukázať, že sú lineárne nezávislé. Ich Wronského determinant je
\begin{equation}
    \begin{aligned}
        W(t) &= \begin{vmatrix}
            e^{st} & t e^{st} \\
            s e^{st} & e^{st} + s t e^{st}
        \end{vmatrix} \\
        &= e^{st} \left( e^{st} + s t e^{st} \right) - t e^{st} s e^{st} \\
        &= e^{2st} + st e^{2st} - st  e^{2st} \\
        &= e^{2st} \neq 0
    \end{aligned}
\end{equation}
čím sme ukázali, že sú lineárne nezávislé. Všeobecné riešenie dif. rovnice \eqref{e01} v tomto prípade je v tvare
\begin{equation}
    y(t) = c_1 e^{st} + c_2 t e^{st}
\end{equation}





\subsection{Prípad: dva komplexne združené korene}

Nech charakteristický polynóm má dva komplexne združené korene
\begin{subequations}
    \begin{align}
        s_1 &= a + j b \\
        s_2 &= a - j b
    \end{align}
\end{subequations}
kde $a, b \in \mathbb R$ a $j$ je imaginárna jednotka.

V tomto prípade funkcie $e^{(a+jb)t}$ a $e^{(a-jb)t}$ nie sú reálne funkcie. S využitím Eulerovho vzťahu
\begin{subequations}
    \begin{align}
        \begin{split}
            e^{s_1t} = e^{at}e^{jbt} &= e^{at} \left( \cos(bt) + j \sin(bt) \right) \\
            &= e^{at} \cos(bt) + j e^{at} \sin(bt)
        \end{split} \\
        \begin{split}
            e^{s_2t} = e^{at}e^{-jbt} &= e^{at} \left( \cos(bt) - j \sin(bt) \right) \\
            &= e^{at} \cos(bt) - j e^{at} \sin(bt)
        \end{split} 
    \end{align}
\end{subequations}
Uvedené funkcie sú teda lineárnou kombináciou funkcií
\begin{subequations}
    \begin{align}
        y_{f1}(t) &= e^{at} \cos(bt) \\
        y_{f2}(t) &= e^{at} \sin(bt)
    \end{align}
\end{subequations}
Ukážme, že $y_{f1}(t)$ a $y_{f2}(t)$ sú riešeniami dif. rovnice \eqref{e01}, a že sú lineárne nezávislé.

Keďže $s_1 = a + jb$ je koreňom charakteristického polynómu, tak
\begin{subequations}
    \begin{align}
        \left( a + jb \right)^2 + a_1 \left( a + jb \right) + a_0 &= 0 \\
        a^2 + 2 a j b - b^2 + a_1 a + a_1 j b + a_0 &= 0
    \end{align}
\end{subequations}
odkiaľ
\begin{subequations}
    \begin{align}
        j \left( 2 a b + a_1 b \right) &= 0  \label{mv1} \\
        a^2 - b^2 + a_1a + a_0 &= 0 \label{mv2}
    \end{align}
\end{subequations}

Vypočítajme derivácie $y_{f1}(t)$:
\begin{subequations}
    \begin{align}
        \begin{split}
            \dot y_{f1}(t) &= e^{at} a \cos(bt) + e^{at}  \left( -\sin(bt) \right) b \\
            &= a e^{at} \cos(bt) - b e^{at} \sin(bt) 
        \end{split} \\
        \begin{split}
            \ddot y_{f1}(t) &= a \left( e^{at} a \cos(bt) - e^{at} b \sin(bt) \right)
            - b \left( e^{at} a \sin(bt) + e^{at} b \cos(bt) \right) \\
            &= a^2 e^{at} \cos(bt) - a b e^{at} \sin(bt) - a b e^{at} \sin(bt) - b^2 e^{at} \cos(bt) \\
            &= \left( a^2 - b^2  \right) e^{at} \cos(bt) - 2 a b e^{at} \sin(bt)
        \end{split}
    \end{align}
\end{subequations}

Dosadením do dif. rovnice \eqref{e01}
\begin{subequations}
    \begin{align}
        \begin{split}
            &\phantom{+}  \left( a^2 - b^2  \right) e^{at} \cos(bt) - 2 a b e^{at} \sin(bt) 
            % \\&+ 
            +
            a_1 \left( a e^{at} \cos(bt) - b e^{at} \sin(bt) \right) 
            \\&+
            a_0 e^{at} \cos(bt) = 0 
        \end{split} \\
        \begin{split}
            &\phantom{+}  \left( a^2 - b^2  \right) e^{at} \cos(bt) - 2 a b e^{at} \sin(bt) 
            % \\&+
            +
            a_1 a e^{at} \cos(bt) - a_1 b e^{at} \sin(bt) 
            \\&+
            a_0 e^{at} \cos(bt) = 0 
        \end{split} \\
        \begin{split}
            &\phantom{+}  \left( a^2 - b^2  + a_1 a + a_0 \right) e^{at} \cos(bt) 
            - 
            \left( 2 a b + a_1 b \right) e^{at} \sin(bt) = 0
        \end{split} \label{drf1}
    \end{align}
\end{subequations}
Vieme, že platí \eqref{mv1} a \eqref{mv2} a teda platí aj \eqref{drf1}. Analogicky je možné ukázať, že aj $y_{f2}(t)$ je riešením dif. rovnice \eqref{e01}. Ostáva ukázať, že tieto riešenia sú navzájom lineárne nezávislé. Ich Wronského determinant je
\begin{equation}
    \begin{aligned}
        W(t) &= \begin{vmatrix}
            e^{at} \cos(bt) & e^{at} \sin(bt) \\
            a e^{at} \cos(bt) - b e^{at} \sin(bt) & a e^{at} \sin(bt) + b e^{at} \cos(bt)
        \end{vmatrix} \\
        &= e^{at} \cos(bt) \left( a e^{at} \sin(bt) + b e^{at} \cos(bt) \right) 
        - e^{at} \sin(bt) \left( a e^{at} \cos(bt) - b e^{at} \sin(bt) \right) \\
        &= e^{2at} \left( a \cos(bt) \sin(bt) + b \cos^2(bt) - a \cos(bt) \sin(bt) + b \sin^2(bt) \right) \\
        &= e^{2at} (b \cos^2(bt) + b \sin^2(bt)) \\
        &= e^{2at} b \neq 0
    \end{aligned}
\end{equation}
čím sme ukázali, že sú lineárne nezávislé. Všeobecné riešenie dif. rovnice \eqref{e01} v tomto prípade je v tvare
\begin{equation}
    y(t) = c_1 e^{at} \cos(bt) + c_2 e^{at} \sin(bt)
\end{equation}



\nocite{*}

\printbibliography[title={Referencie a ďalšia literatúra}]

% -----------------------------------------------------------------------------

\end{document}

% -----------------------------------------------------------------------------