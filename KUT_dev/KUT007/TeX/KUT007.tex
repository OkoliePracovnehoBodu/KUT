\documentclass[a4paper, 10pt, ]{article}

\input{./COMMONFILES/preamble.tex}

% -----------------------------------------------------------------------------

\def\oznacenieCelku{Kolekcia učebných textov}

% -----------------------------------------------------------------------------


\def\KUTporadoveCislo{007}

\def\oznacenieVerzie{v0.9}
% \def\oznacenieVerzie{\phantom{v1.0}}

\def\mesiacRok{júl 2024}

\def\authorslabel{MT}





% -----------------------------------------------------------------------------

\begin{document}

% -----------------------------------------------------------------------------
% Uvodny nadpis

\noindent
\parbox[t][18mm][c]{0.3\textwidth}{%
\raisebox{-0.9\height}{%
\phantom{.}\includegraphics[height=18mm]{./COMMONFILES/URKFEIlogo.pdf}%
}%
}%
\parbox[t][18mm][c]{0.7\textwidth}{%
\raggedleft

\sffamily
\fontsize{16pt}{18pt}
\fontseries{sbc}
\selectfont

\noindent
\textcolor[rgb]{0.75, 0.75, 0.75}{\textls[25]{\oznacenieCelku}}
}%

\noindent
\parbox[t][16mm][b]{0.5\textwidth}{%
\raggedright

\color{Gray}
\sffamily

\fontsize{12pt}{12pt}
\selectfont
\mesiacRok

\fontsize{6pt}{10pt}
\selectfont
github.com/PracovnyBod/KUT

\fontsize{8pt}{10pt}
\selectfont
\authorslabel




}%
\parbox[t][16mm][b]{0.5\textwidth}{%
\raggedleft

\sffamily

\fontsize{6pt}{6pt}
\selectfont

\textcolor[rgb]{0.68, 0.68, 0.68}{\oznacenieVerzie}


\fontsize{14pt}{14pt}
\selectfont

\bfseries

\includegraphics[height=12pt]{./COMMONFILES/KUT_logo_v0.1.pdf}%
{%
\textls[-50]{\KUTporadoveCislo}
}%
}%

% -----------------------------------------------------------------------------




\vspace{6mm}

% ---------------------------------------------
\sffamily
\bfseries
\fontsize{18pt}{21pt}
\selectfont

\begin{flushleft}
	O schematickom znázornení\\ dynamického systému
\end{flushleft}

\bigskip

% -----------------------------------------------------------------------------
\normalsize
\normalfont
% -----------------------------------------------------------------------------












\noindent
\lettrine[lines=1, nindent=1pt, loversize=0.0]{D}{ynamický} 
 systém je možné, samozrejme, opísať diferenciálnou rovnicou a túto potom použiť na ďalšiu analýzu. Dynamický systém, alebo diferenciálnu rovnicu, ktorá ho opisuje, je možné znázorniť aj graficky \emph{blokovou schémou}. Takou, ktorá prípadne umožňuje aj opačný postup, teda schéma určuje diferenciálnu rovnicu. Výsledná bloková schéma taktiež umožňuje ďalšiu analýzu dynamického systému.


\section{Prvky blokovej schémy}

 Typicky sa v rámci takejto schémy používajú nasledovné prvky a bloky.

 \paragraph{Signál}

 \begin{center}

    \vbox{%
    \makebox[\textwidth][c]{%
	\input{../SVG/schB_signal.pdf_tex}
	}

	\figcaption{Signál v blokovej schéme.}
    \label{schB_signal.pdf}
    }

\end{center}

\noindent
Signál je reprezentovaný čiarou so šípkou, ktorá určuje smer prenosu informácie. Pri čiare je uvedené označenie signálu.


\paragraph{Zosilňovač}

\begin{center}

    \vbox{%
    \makebox[\textwidth][c]{%
	\input{../SVG/schB_gain.pdf_tex}
	}

	\figcaption{Zosilňovač v blokovej schéme.}
    \label{schB_gain.pdf}
    }

\end{center}

\noindent
Ide o blok, ktorý vo všeobecnosti zosilňuje signál na vstupe tohto bloku. Samozrejme, môže ísť aj o „zoslabenie“. Inými slovami, tento blok vynásobí hodnotu vstupného signálu $u(t)$ hodnotou parametra $a$ a výstupom je signál $y(t)$. Matematicky zapísané
\begin{equation}
    y(t) = a \,  u(t)
\end{equation}
Parameter $a$ môže mať ľubovolnú hodnotu, môže byť menší ako $1$ (zoslabenie) alebo záporný.


\paragraph{Sumátor}

\begin{center}

    \vbox{%
    \makebox[\textwidth][c]{%
	\input{../SVG/schB_sumator.pdf_tex}
	}

	\figcaption{Sumátor v blokovej schéme.}
    \label{schB_sumator.pdf}
    }

\end{center}

\noindent
Realizuje sčítanie hodnôt dvoch alebo viacerých signálov. Prípadne odčítanie. Príklad matematického zápisu:
\begin{equation}
    y(t) = u_1(t) + u_2(t)
\end{equation}

\paragraph{Integrátor}

\begin{center}

    \vbox{%
    \makebox[\textwidth][c]{%
    \input{../SVG/schB_integrator.pdf_tex}
    }

    \figcaption{Integrátor v blokovej schéme.}
    \label{schB_integrator.pdf}
    }

\end{center}

\noindent
Predstavuje časovú integráciu vstupného signálu v zmysle matematického zápisu
\begin{equation}
    y(t) = \int u(t) \, \text{d}t \qquad y(0) = y_0
\end{equation}
kde $y_0$ je začiatočná hodnota signálu $y(t)$, teda začiatočná podmienka.




\paragraph{Derivácia}

Pre úplnosť uvedieme aj blok realizujúci časovú deriváciu signálu. Je opakom integrátora. Tento blok sa však v praxi používa menej často. Dôvodom je, že implementovať časovú deriváciu je v praxi

\begin{center}

    \vbox{%
    \makebox[\textwidth][c]{%
    \input{../SVG/schB_derivacia.pdf_tex}
    }

    \figcaption{Časová derivácia v blokovej schéme.}
    \label{schB_derivacia.pdf}
    }

\end{center}







\section{Príklady}

\subsection{Príklad postupu}

Uvažujme dynamický systém daný diferenciálnou rovnicou v tvare
\begin{equation} \label{eq:HDR1R}
    \dot y(t) + a y(t) = 0 \qquad y(0) = y_0
\end{equation}
Postup pre zostavenie blokovej schémy dynamického systému môže byť nasledovný.

Rovnica \eqref{eq:HDR1R} je prvého rádu a neznámou je časová funkcia $y(t)$. Prepíšme rovnicu \eqref{eq:HDR1R} tak, aby na ľavej strane bola len najvyššia derivácia neznámej, v tomto prípade signál $\dot y(t)$. Teda
\begin{equation} \label{eq:HDR1Rs}
    \dot y(t) = - a y(t)
\end{equation}
Rovnica v tomto tvare je východiskom pre zostavenie blokovej schémy. Je totiž zrejmé, že signál $\dot y(t)$  existuje. Inými slovami, to, čo určite máme k dispozícii je signál $\dot y(t)$. Ak by tento signál neexistoval, tak vlastne rovnica \eqref{eq:HDR1Rs} by bola nezmyslom. V schéme teda máme k dispozícii signál $\dot y(t)$.
\begin{center}

    \vbox{%
    \makebox[\textwidth][c]{%
    \input{../SVG/schB_pr0_k1.pdf_tex}
    }

    \vspace{-5mm}

    \figcaption{Bloková schéma rovnice \eqref{eq:HDR1R}, krok prvý.}
    \label{schB_pr0_k1.pdf}
    }

\end{center}
Rovnica v tvare \eqref{eq:HDR1Rs} tiež priamo ukazuje, že signál $\dot y(t)$ je to isté ako výraz $- a y(t)$. Vieme zostaviť blokovú schému tohto výrazu? Ide zjavne o zosilňovač so zosilnením $-a$, ktorý má na vstupe signál $y(t)$. Pridajme do blokovej schémy:
\begin{center}

    \vbox{%
    \makebox[\textwidth][c]{%
    \input{../SVG/schB_pr0_k2.pdf_tex}
    }

    \figcaption{Bloková schéma rovnice \eqref{eq:HDR1R}, krok druhý.}
    \label{schB_pr0_k2.pdf}
    }

\end{center}
Keďže doslova $\dot y(t) = - a y(t)$, tak
\begin{center}

    \vbox{%
    \makebox[\textwidth][c]{%
    \input{../SVG/schB_pr0_k3.pdf_tex}
    }

    \figcaption{Bloková schéma rovnice \eqref{eq:HDR1R}, krok tretí.}
    \label{schB_pr0_k3.pdf}
    }

\end{center}
Pripomeňme, že signál $\dot y(t)$ takpovediac existuje, je k dispozícii. Samotný signál $y(t)$ však nie je k dispozícii. Je potrebné ho vytvoriť z toho, čo už k dispozícii je. Je zrejmé, že signál $y(t)$ je možné získať integrovaním $\dot y(t)$, teda
\begin{center}

    \vbox{%
    \makebox[\textwidth][c]{%
    \input{../SVG/schB_pr0_k4.pdf_tex}
    }

    \figcaption{Bloková schéma rovnice \eqref{eq:HDR1R}, krok štvrtý.}
    \label{schB_pr0_k4.pdf}
    }

\end{center}
pričom integrátor musí mať začiatočnú podmienku $y(0) = y_0$ (podľa \eqref{eq:HDR1R}).
\begin{center}

    \vbox{%
    \makebox[\textwidth][c]{%
    \input{../SVG/schB_pr0_k4b.pdf_tex}
    }

    \figcaption{}
    \label{schB_pr0_k4b.pdf}
    }

\end{center}
Napokon
\begin{center}

    \vbox{%
    \makebox[\textwidth][c]{%
    \input{../SVG/schB_pr0_k5.pdf_tex}
    }

    \figcaption{Bloková schéma rovnice \eqref{eq:HDR1R}.}
    \label{schB_pr0_k5.pdf}
    }

\end{center}
je bloková schéma dynamického systému, ktorá zodpovedá diferenciálnej rovnici \eqref{eq:HDR1R}.




\subsection{Príklad 1}

Uvažujme dynamický systém daný diferenciálnou rovnicou v tvare
\begin{equation} \label{eq:DR1R}
    \dot y(t) + a y(t) = b u(t) \qquad y(0) = y_0
\end{equation}
kde $a$, $b$ sú konštanty a $u(t)$ je známy vstupný signál.

Rovnicu \eqref{eq:DR1R} prepíšme tak, aby na ľavej strane bola len najvyššia derivácia neznámej, teda signál $\dot y(t)$. Teda
\begin{equation} \label{eq:DR1Rs}
    \dot y(t) = - a y(t) + b u(t)
\end{equation}
Na začiatku máme k dispozícii signál $\dot y(t)$, teda
\begin{center}

    \vbox{%
    \makebox[\textwidth][c]{%
    \input{../SVG/schB_pr1_k1.pdf_tex}
    }

    \vspace{-15mm}

    \figcaption{Bloková schéma rovnice \eqref{eq:DR1R}, krok prvý.}
    \label{schB_pr1_k1.pdf}
    }

\end{center}
Signál $\dot y(t)$ je súčtom dvoch iných signálov.
\begin{center}

    \vbox{%
    \makebox[\textwidth][c]{%
    \input{../SVG/schB_pr1_k2.pdf_tex}
    }

    % \vspace{-5mm}

    \figcaption{Bloková schéma rovnice \eqref{eq:DR1R}, krok druhý.}
    \label{schB_pr1_k2.pdf}
    }

\end{center}
Prvý signál získame zosilnením signálu $y(t)$ zosilňovačom s~parametrom $-a$.
\begin{center}

    \vbox{%
    \makebox[\textwidth][c]{%
    \input{../SVG/schB_pr1_k3.pdf_tex}
    }

    \figcaption{Bloková schéma rovnice \eqref{eq:DR1R}, krok tretí.}
    \label{schB_pr1_k3.pdf}
    }

\end{center}
Druhý získame zosilnením známeho (dostupného) signálu $u(t)$ zosilňovačom s~parametrom $b$.
\begin{center}

    \vbox{%
    \makebox[\textwidth][c]{%
    \input{../SVG/schB_pr1_k4.pdf_tex}
    }

    \figcaption{Bloková schéma rovnice \eqref{eq:DR1R}, krok štvrtý.}
    \label{schB_pr1_k4.pdf}
    }

\end{center}
Signál $y(t)$ v tomto kroku však nie je dostupný, je potrebné ho vytvoriť z toho, čo už k dispozícii je. Signál $y(t)$ je možné získať integrovaním signálu $\dot y(t)$.
\begin{center}

    \vbox{%
    \makebox[\textwidth][c]{%
    \input{../SVG/schB_pr1_kf.pdf_tex}
    }

    \figcaption{Bloková schéma rovnice \eqref{eq:DR1R}.}
    \label{schB_pr1_kf.pdf}
    }

\end{center}
Integrátor musí mať začiatočnú podmienku $y(0) = y_0$ (podľa \eqref{eq:DR1R}).

















\subsection{Príklad 2}

Uvažujme dynamický systém daný diferenciálnou rovnicou v tvare
\begin{equation} \label{eq:DR2R}
    \ddot y(t) + a_1 \dot y(t) + a_0 y(t) = b_0 u(t) \qquad y(0) = y_0 \qquad \dot y(0) = z_0
\end{equation}
kde $a_0$, $a_1$, $b_0$ sú konštanty a $u(t)$ je známy vstupný signál.

Rovnicu \eqref{eq:DR2R} prepíšme tak, aby na ľavej strane bola len najvyššia derivácia neznámej, teda signál $\ddot y(t)$. Teda
\begin{equation} \label{eq:DR2Rs}
    \ddot y(t) = - a_1 \dot y(t) - a_0 y(t) + b_0 u(t)
\end{equation}
Na začiatku máme k dispozícii signál $\ddot y(t)$, teda
\begin{center}

    \vbox{%
    \makebox[\textwidth][c]{%
    \input{../SVG/schB_pr2_k1.pdf_tex}
    }

    \vspace{-30mm}

    \figcaption{Bloková schéma rovnice \eqref{eq:DR2R}, krok prvý.}
    \label{schB_pr2_k1.pdf}
    }

\end{center}
Signál $\ddot y(t)$ je v podstate súčtom troch iných signálov.
\begin{center}

    \vbox{%
    \makebox[\textwidth][c]{%
    \input{../SVG/schB_pr2_k2.pdf_tex}
    }

    \vspace{-5mm}

    \figcaption{Bloková schéma rovnice \eqref{eq:DR2R}, krok druhý.}
    \label{schB_pr2_k2.pdf}
    }

\end{center}
Prvý signál získame zosilnením signálu $\dot y(t)$ zosilňovačom so zosilnením $a_1$. Signál $\dot y(t)$ je možné získať integrovaním signálu $\ddot y(t)$.
\begin{center}

    \vbox{%
    \makebox[\textwidth][c]{%
    \input{../SVG/schB_pr2_k3.pdf_tex}
    }

    \vspace{-5mm}

    \figcaption{Bloková schéma rovnice \eqref{eq:DR2R}, krok tretí.}
    \label{schB_pr2_k3.pdf}
    }

\end{center}
Druhý signál získame zosilnením signálu $y(t)$ zosilňovačom so zosilnením $a_0$. Signál $y(t)$ je možné získať integrovaním signálu $\dot y(t)$.
\begin{center}

    \vbox{%
    \makebox[\textwidth][c]{%
    \input{../SVG/schB_pr2_k4.pdf_tex}
    }

    \figcaption{Bloková schéma rovnice \eqref{eq:DR2R}, krok štvrtý.}
    \label{schB_pr2_k4.pdf}
    }

\end{center}
Tretí signál získame zosilnením známeho (dostupného) signálu $u(t)$ zosilňovačom so zosilnením $b_0$. 
\begin{center}

    \vbox{%
    \makebox[\textwidth][c]{%
    \input{../SVG/schB_pr2_kf.pdf_tex}
    }

    \figcaption{Bloková schéma rovnice \eqref{eq:DR2R}.}
    \label{schB_pr2_kf.pdf}
    }

\end{center}
Príslušné integrátori vo výslednej schéme musia mať začiatočné podmienky $y(0) = y_0$ a $\dot y(0) = z_0$ (podľa \eqref{eq:DR2R}).











% -----------------------------------------------------------------------------

\end{document}

% -----------------------------------------------------------------------------