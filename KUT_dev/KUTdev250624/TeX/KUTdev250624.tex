\documentclass[a4paper, 10pt, ]{article}

\input{./COMMONFILES/preamble.tex}

% -----------------------------------------------------------------------------

\def\oznacenieCelku{Kolekcia učebných textov}

% -----------------------------------------------------------------------------


\def\KUTporadoveCislo{dev250624}

\def\oznacenieVerzie{v0.9}
% \def\oznacenieVerzie{\phantom{v1.0}}

\def\mesiacRok{jún 2025}

\def\authorslabel{MT}






% -----------------------------------------------------------------------------

\begin{document}

% -----------------------------------------------------------------------------
% Uvodny nadpis

\noindent
\parbox[t][18mm][c]{0.3\textwidth}{%
\raisebox{-0.9\height}{%
\phantom{.}\includegraphics[height=18mm]{./COMMONFILES/URKFEIlogo.pdf}%
}%
}%
\parbox[t][18mm][c]{0.7\textwidth}{%
\raggedleft

\sffamily
\fontsize{16pt}{18pt}
\fontseries{sbc}
\selectfont

\noindent
\textcolor[rgb]{0.75, 0.75, 0.75}{\textls[25]{\oznacenieCelku}}
}%

\noindent
\parbox[t][16mm][b]{0.5\textwidth}{%
\raggedright

\color{Gray}
\sffamily

\fontsize{12pt}{12pt}
\selectfont
\mesiacRok

\fontsize{6pt}{10pt}
\selectfont
github.com/OkoliePracovnehoBodu/KUT

\fontsize{8pt}{10pt}
\selectfont
\authorslabel




}%
\parbox[t][16mm][b]{0.5\textwidth}{%
\raggedleft

\sffamily

\fontsize{6pt}{6pt}
\selectfont

\textcolor[rgb]{0.68, 0.68, 0.68}{\oznacenieVerzie}


\fontsize{14pt}{14pt}
\selectfont

\bfseries

\includegraphics[height=12pt]{./COMMONFILES/KUT_logo_v0.1.pdf}%
{%
\textls[-50]{\KUTporadoveCislo}
}%
}%

% -----------------------------------------------------------------------------




\vspace{6mm}

% ---------------------------------------------
\sffamily
\bfseries
\fontsize{18pt}{21pt}
\selectfont

\begin{flushleft}
    Laboratórne zariadenie LMOT:\\ statické vlastnosti
\end{flushleft}

\bigskip

% -----------------------------------------------------------------------------
\normalsize
\normalfont
% -----------------------------------------------------------------------------

\lstset{style=mystyle}










\noindent
\lettrine[lines=1, nindent=1pt, loversize=0.0]{C}{ieľom} 
textu je opis laboratórneho dynamického systému LMOT z hľadiska jeho statických vlastností.


\section{Opis dynamického systému LMOT}

LMOT je laboratórne zariadenie predstavujúce reálny dynamický systém. Pozostáva z malého jednosmerného motora, tachodynama, ktoré je na spoločnom hriadeli s~motorom, a z elektronických obvodov, ktoré zabezpečujú napájanie motora. Elektronickymi obvodmi sú tiež dané dominantné statické a dynamické vlastnosti výsledného systému. Do istej miery je možné tieto vlastnosti meniť manuálnym nastavením príslušného potenciometra.

Systém má jeden vstupný signál a jeden výstupný signál. Výstupný signál je priamo úmerný uhlovej rýchlosti jednosmerného motora, ktorá je snímaná tachodynamom. Vstupný signál ovláda napájanie motora.

Polohou potenciometra je v podstate daná prevádzková podmienka zariadenia. K~dispozícii je signál zodpovedajúci polohe potenciometra a teda tým je k dispozícii informácia o prevádzkovej podmienke systému.

LMOT (čítaj \emph{elmot}) je akronym pre „laboratórny motorček“, prípadne pre „little motor“.



\section{Meranie prevodovej charakteristiky}

V kontexte statických vlastností systému má vo všeobecnosti význam hovoriť o~prevodovej charakteristike systému. Prevodová charakteristika je závislosť ustálených hodnôt výstupného signálu systému od ustálených hodnôt vstupného signálu systému.

Je zrejmé, že prevodová charakteristika sa týka systémov s prívlastkom statické, teda takých, ktoré nie sú astatické.

Prevodová charakteristika, niekde sa nazýva aj statická charakteristika, teda charakterizuje systém len v~ustálených stavoch. Neobsahuje informáciu o dynamike systému.



\subsection{Návrh merania}

Z opisu predmetného dynamického systému vyplýva, že systém má jeden výstupný signál, jeden vstupný signál a manuálne nastaviteľnú prevádzkovú podmienku.

\paragraph{Rozsahy a jednotky signálov}

Vstupný a výstupný signál nadobúdajú hodnoty v~rozsahu $0$ až $10$ pričom ide o~napäťové signály vo voltoch [V].

Prevádzková podmienka systému sa nastavuje manuálne otáčaním potenciometra. Signál o polohe potenciometra nadobúda hodnoty v rozsahu $0$ [V] až $10$ [V].

\paragraph{Voľba ustálených hodnôt vstupov}

O predmetnom systéme je známe, že výstup systému sa ustáli vždy ak sú vstupy systému ustálené. Pre vyšetrovanie ustálených stavov je teda možné využiť celý rozsah vstupného signálu a celý rozsah prevádzkových podmienok.

Návrh uvažuje ustálené hodnoty vstupného signálu uvedené v tabuľke~\ref{tab:ustalene_hodnoty_vstupneho_signalu} a zároveň ustálené hodnoty reprezentujúce prevádzkové podmienky podľa tabuľky \ref{tab:ustalene_hodnoty_podmienok}.


\begin{center}

\vspace{-10pt}    
    
\tabcaption{Ustálené hodnoty vstupného signálu [V]}
\label{tab:ustalene_hodnoty_vstupneho_signalu}

\lstyle

\begin{tabular*}{\textwidth}{@{ \extracolsep{\fill}} ccccccccccc}
\toprule
0 & 1 & 2 & 3 & 4 & 5 & 6 & 7 & 8 & 9 & 10 \\
\bottomrule
\end{tabular*}


\tabcaption{Ustálené hodnoty signálu o prevádkových podmienkach [V]}
\label{tab:ustalene_hodnoty_podmienok}

\lstyle

\begin{tabular*}{\textwidth}{@{ \extracolsep{\fill}} ccccccccccc}
\toprule
0 & 1 & 2 & 3 & 4 & 5 & 6 & 7 & 8 & 9 & 10 \\
\bottomrule
\end{tabular*}

\end{center}


\paragraph{Voľba časového intervalu pre ustálenie výstupu systému}

Empirické skúsenosti so systémom ukazujú, že z praktického hľadiska sa systém ustáli do $15$ sekúnd po zmene na vstupe systému. Ukazuje sa však aj náchylnosť systému k poruchám spôsobeným zväčša mechanickými nedostatkami a vibráciami zrejme spôsobujúcimi zmeny trenia v mechanických častiach systému. Pre pozorovanie a~vyhodnotenie vplyvu týchto porúch v ustálenom stave je časový interval pre ustálenie zvolený na $120$ sekúnd.



\paragraph{Postup merania}

Vzhľadom na uvedené voľby ustálených hodnôt a časového intervalu návrh predpokladá nasledovný postup.

\begin{enumerate}[leftmargin=0pt, labelsep=4mm, itemsep=0pt]
    \item Manuálne nastavenie prevádzkových podmienok na hodnotu z tabuľky~\ref{tab:ustalene_hodnoty_podmienok}.
    \item Postupná zmena vstupného signálu na hodnoty z tabuľky~\ref{tab:ustalene_hodnoty_vstupneho_signalu} so zvoleným časovým intervalom. Takúto postupnú zmenu vyjadruje nasledujúca tabuľka \ref{tab:postupna_zmena_vstupneho_signalu}.
    
\end{enumerate}


\begin{center}

\vspace{-10pt}    

\tabcaption{Postupná zmena vstupného signálu}
\label{tab:postupna_zmena_vstupneho_signalu}

\lstyle

\begin{tabular*}{\textwidth}{@{ \extracolsep{\fill}} cc}
\toprule
Čas zmeny vstupného signálu [s] & Hodnota vstupného signál [V] \\
\midrule
0 & 0 \\
120 & 1 \\
240 & 2 \\
360 & 3 \\
480 & 4 \\
600 & 5 \\
720 & 6 \\
840 & 7 \\
960 & 8 \\
1080 & 9 \\
1200 & 10 \\
\bottomrule
\end{tabular*}

\end{center}

Celková dĺžka merania je teda $1200 + 120 = 1320$ sekúnd a počas tejto doby sú prevádzkové podmienky konštantné.



\section{Získané dáta}

Príklad získaných dát podľa uvedeného postupu je znázornený na nasledujúcom obrázku \ref{fj_01_data_pot0}.  


% \begin{center}
    \begin{figure}[t]

    \vbox{%
        \makebox[\textwidth][c]{%
        \includegraphics{fj_01_data_pot0_panel_1.pdf}
        }

        \makebox[\textwidth][c]{%
        \includegraphics{fj_01_data_pot0_panel_3.pdf}
        }

        \makebox[\textwidth][c]{%
        \includegraphics{fj_01_data_pot0_panel_2.pdf}
        }

        % \figcaption{ 
        \caption{
            Získané dáta pre vstupný signál a výstupný signál systému LMOT pri manuálne nastavenej prevádzkovej podmienke $0$ [V].
        }
        \label{fj_01_data_pot0}
    }%vbox

    \end{figure}
% \end{center}

Získané dáta pre ostatné prevádzkové podmienky sú uvedené v~prílohách tohto textu. Vizualizované sú tu tak dáta pre všetky prevádzkové podmienky, ktoré sú v~tabuľke~\ref{tab:ustalene_hodnoty_podmienok}.












\section{Spracovanie dát pre prevádzkovú podmienku 0 [V]}

\subsection{Vyznačenie ustálených úsekov}

Zo získaných dát na obrázku \ref{fj_01_data_pot0} je potrebné vyňať úseky, počas ktorých sú výstupný a~vstupný signál ustálené. V tomto prípade nech za ustálený stav sa považuje posledných 30 \% zvoleného časového intervalu pre ustálenie, teda posledných 36 sekúnd z celkových 120 sekúnd. Takto vyňaté úseky sú znázornené na obrázku \ref{fj_02_data_pot0}.

% \begin{center}
    \begin{figure}[!t]

    \vbox{%
        \makebox[\textwidth][c]{%
        \includegraphics{fj_02_data_pot0_panel_1.pdf}
        }

        \makebox[\textwidth][c]{%
        \includegraphics{fj_02_data_pot0_panel_3.pdf}
        }

        \vspace{-6pt}

        % \figcaption{ 
        \caption{
            % 
        }
        \label{fj_02_data_pot0}
    }%vbox

\end{figure}
% \end{center}







\subsection{Nameraná prevodová charakteristika}

Vyňaté úseky je možné zlúčiť do jedného dátového súboru. Obsahuje len hodnoty vstupného signálu a výstupného signálu, pre ktoré je možné konštatovať, že zodpovedajú ustálenému stavu. Takéto dáta charakterizujú systém v ustálených stavoch.

Grafické znázornenie uvedených dát tak, že na x-ovej osi sú hodnoty vstupného signálu v ustálenom stave a na y-ovej osi sú prislúchajúce hodnoty výstupného signálu v~ustálenom stave, je vlastne grafickým vyjadrením nameranej prevodovej charakteristiky. Nameraná prevodová charakteristika je znázornená na obrázku \ref{fj_03_data_pot0_panel_1}.

Pre lepšiu ilustráciu, že uvedený dataset neobsahuje takpovediac len niekoľko bodov, ale že každý ustálený stav je reprezentovaný viacerými hodnotami, je na obrázku \ref{fj_03_data_pot0_panel_2} zobrazený detail pre zvolenú hodnotu vstupného signálu.

% \begin{center}
    \begin{figure}[!t]

    \vbox{%
        \makebox[\textwidth][c]{%
        \includegraphics{fj_03_data_pot0_panel_1.pdf}
        }

        \vspace{-6pt}

        % \figcaption{ 
        \caption{
            Nameraná prevodová charakteristika pre prevádzkovú podmienku $0$ [V]. 
        }
        \label{fj_03_data_pot0_panel_1}
    }%vbox

\end{figure}
% \end{center}


% \begin{center}
    \begin{figure}[!t]

    \vbox{%
        \makebox[\textwidth][c]{%
        \includegraphics{fj_03_data_pot0_panel_2.pdf}
        }

        \vspace{-6pt}

        % \figcaption{ 
        \caption{
            % 
        }
        \label{fj_03_data_pot0_panel_2}
    }%vbox

\end{figure}
% \end{center}




\subsection{Aproximácia prevodovej charakteristiky}
\label{sec:Aproximácia prevodovej charakteristiky}

Nameraná prevodová charakteristika znázornená na obrázku \ref{fj_03_data_pot0_panel_1} zachytáva vzťah medzi vstupným a výstupným signálom v konkrétnych ustálených stavoch. Pre pokrytie celého rozsahu vstupného a výstupného signálu je účelné identifikovať model, ktorý by aproximoval nameranú prevodovú charakteristiku. 

Grafické znázornenie nameranej prevodovej charakteristiky na obrázku \ref{fj_03_data_pot0_panel_1}  ukazuje, že od hodnoty vstupného signálu $7$ [V] a viac je výstupná hodnota saturovaná na maxime (v podstate $10$ [V]). 

Vstupnú veličinu označme ako $u$ a výstupnú veličinu ako $y$. Závislosť výstupnej veličiny od vstupnej veličiny potom je 
\begin{equation}
    y = f(u)
\end{equation}
kde $f$ je funkcia, ktorá vyjadruje prevodovú charakteristiku systému.



Pre stanovenie modelu rozdeľme prevodovú charakteristiku na dve časti. Hľadáme funkciu $f_1$, ktorá bude aproximovať prevodovú charakteristiku pre hodnoty vstupného signálu $u$ v rozsahu $0 \leq u \leq 7$ [V] a funkciu $f_2$, ktorá bude aproximovať prevodovú charakteristiku pre hodnoty vstupného signálu $u$ v rozsahu $7 < u \leq 10$ [V].

Funkcia $f_2$ je v tomto prípade konštantná, pretože výstupný signál je saturovaný na hodnote $10$ [V]. Výstup modelu $\hat y$ na príslušnom rozsahu teda je
\begin{equation}
    \hat y = f_2(u) = 10 \quad \text{pre } 7 < u \leq 10
\end{equation}

Na úseku $0 \leq u \leq 7$ [V], na základe grafického zobrazenia nameranej prevodovej charakteristiky, je možné usúdiť, že hľadaná závislosť je blízka priamke. Preto nech funkcia $f_1$ je daná polynómom prvého stupňa, teda
\begin{equation}
    f_1(u) = \Theta_1 u + \Theta_0
\end{equation}
kde $\Theta_1$ a $\Theta_0$ sú parametre, ktoré je potrebné určiť.

Parametre je možné určiť metódou najmenších štvorcov, čo v tomto prípade znamená minimalizovať sumu štvorcov odchýlok medzi nameranými hodnotami výstupného signálu a hodnotami vypočítanými z modelu. Formálne je odchýlka $ e = y - \hat y$. Takúto polynomiálnu aproximáciu bežne implementujú knižnice pre numerické výpočty. V tomto prípade využijeme knižnicu \texttt{NumPy} a jej funkciu \texttt{polyfit}. Výsledkom je 
\begin{equation}
    f_1(u) = 1,399\ u - 0,1196
\end{equation}
a táto funkcia je znázornená na obrázku \ref{fj_04_data_pot0_panel_1}.

% \begin{center}
    \begin{figure}[!t]

    \vbox{%
        \makebox[\textwidth][c]{%
        \includegraphics{fj_04_data_pot0_panel_1.pdf}
        }

        \vspace{-6pt}

        % \figcaption{ 
        \caption{
            % 
        }
        \label{fj_04_data_pot0_panel_1}
    }%vbox

\end{figure}
% \end{center}

% \begin{center}
    \begin{figure}[!b]

    \vbox{%
        \makebox[\textwidth][c]{%
        \includegraphics{fj_04_data_pot0_panel_2.pdf}
        }

        \vspace{-6pt}

        % \figcaption{ 
        \caption{
            % 
        }
        \label{fj_04_data_pot0_panel_2}
    }%vbox

\end{figure}
% \end{center}

Ak by sme chceli model vyjadriť ako celok, formálne je možné písať
\begin{equation}
    \hat y = 
    \begin{cases}
        f_1(u) = 1,399\ u - 0,1196 & \text{pre } 0 \leq u \leq 7 \\
        f_2(u) = 10 & \text{pre } 7 < u \leq 10
    \end{cases}
\end{equation}
Graficky je výstup modelu znázornený na obrázku \ref{fj_04_data_pot0_panel_2}.




\subsection{Skript}

Python skript pre realizáciu vyššie uvedeného spracovania dát je dostupný v~súbore \texttt{dataJob\_SSChar.ipynb}. Ide o Jupyter notebook pričom tento skript je v bunke č.~5.


\begin{lstlisting}[language=Python, caption=Súbor \lstinline|../dataJob_SSChar.ipynb cell:05|]
# Spracovanie dát pre prevádzkovú podmienku 0 [V]

data_pot = 'data_pot0' # Oznacenie pre prevadzkovu podmienku 0 [V]


# ----------------------------------------------
# Vykreslenie ustalenych usekov

exec(open('./figjobs/figJob_02.py', encoding='utf-8').read()) # skript je v subore figJob_02.py


# ----------------------------------------------
# Zlucenie usekov do datasetu ustalenych stavov

dataRepoDir = './dataRepo/'

sch_files = [
    f for f in os.listdir(dataRepoDir)
    if f.startswith('SCH') and data_pot in f
]

steadyStateData = np.empty((0, 2))

for sch_file in sch_files:

    sch_data = np.loadtxt(
        dataRepoDir + sch_file,
        delimiter=',',
        skiprows=1,
    )

    steadyStateData = np.vstack([steadyStateData, sch_data[:, 1:3]])

np.savetxt(dataRepoDir + 'ALLSCH_' + data_pot + '_steadyStateData' + '.csv',
    steadyStateData,
    delimiter=',',
    fmt='%.3f',
    header='sigOut,sigIn',
    comments='',
)


# ----------------------------------------------
# Vykreslenie dat ustalenych stavov

exec(open('./figjobs/figJob_03.py', encoding='utf-8').read())

# ----------------------------------------------
# Identifikacia modelu na linearnom useku

tmpMask = (steadyStateData[:, 1] >= 0) & (steadyStateData[:, 1] <= 7.1)

identData = steadyStateData[tmpMask, :]

polyKoef = np.polyfit(
    identData[:, 1],
    identData[:, 0],
    deg=1
)



# plot data pre f_1
plot_u1 = np.arange(0, 7.1, 0.1)
plot_y1 = np.polyval(polyKoef, plot_u1)

# plot data pre f_2
plot_u2 = np.arange(7, 10.1, 0.1)
plot_y2 = 10 * np.ones_like(plot_u2)

exec(open('./figjobs/figJob_04.py', encoding='utf-8').read()) \end{lstlisting}









\section{Spracovanie dát pre všetky prevádzkové podmienky}

Obdobne ako na obr.~\ref{fj_03_data_pot0_panel_1} je možné vizualizovať všetky namerané prevodové charakteristiky pre všetky prevádzkové podmienky, pozri obrázok \ref{fj_05__panel_1}. 

% \begin{center}
    \begin{figure}[!t]

    \vbox{%
        \makebox[\textwidth][c]{%
        \includegraphics{fj_05__panel_1.png}
        }

        \vspace{-6pt}

        % \figcaption{ 
        \caption{
            % 
        }
        \label{fj_05__panel_1}
    }%vbox

\end{figure}
% \end{center}

Na tieto dáta, a teda na skúmaný systém ako celok, sa dá nazerať ako na dáta zodpovedajúce systému s~dvomi vstupmi a jedným výstupom. Prvým vstupom je dosiaľ uvádzaný vstupný signál, druhým vstupom je signál o prevádzkovej podmienke. Dáta reprezentujúce ustálené stavy je potom možné vizualizovať ako funkciu dvoch premenných, pozri obrázok \ref{fj_06__panel_1}.

% \begin{center}
    \begin{figure}[!h]

    \vspace{-4mm}

    \vbox{%
        \makebox[\textwidth][c]{%
        \includegraphics{fj_06__panel_1.png}
        }

        \vspace{-6pt}

        % \figcaption{ 
        \caption{
            % 
        }
        \label{fj_06__panel_1}
    }%vbox

\end{figure}
% \end{center}


\subsection{Aproximácia prevodovej charakteristiky pre každú prevádzkovú podmienku zvlášť}

V princípe je možné zrealizovať aproximáciu prevodovej charakteristiky podľa postupu uvedeného v~časti \ref{sec:Aproximácia prevodovej charakteristiky} pre každú prevádzkovú podmienku zvlášť. 

Vo všeobecnosti je však potrebné zvážiť aj saturáciu výstupného signálu, ktorá sa prejavuje pri nízkych hodnotách vstupného signálu. Napríklad pre prevádzkovú podmienku $5$ [V] je nameraná prevodová charakteristika znázornená na obrázku \ref{fj_04b_data_pot5_panel_2}.

Nech hraničné hodnoty dolnej a hornej saturácie sú $1$ [V] a $9$ [V]. Potom model aproximujúci nameranú prevodovú charakteristiku pre je možné vyjadriť v tvare
\begin{equation}
    \hat y = 
    \begin{cases}
        f_1(u) = \Theta_1\ u - \Theta_0 & \text{pre } 1 \leq u \leq 9 \\
        f_2(u) = 10 & \text{pre } u > 9 \\
        f_3(u) = 0 & \text{pre } u < 1         
    \end{cases}
\end{equation}
pričom v tomto prípade sú parametre $\Theta_1 = 1,185$ a $\Theta_0 = -1,172$.



% \begin{center}
    \begin{figure}[!t]

    \vbox{%
        \makebox[\textwidth][c]{%
        \includegraphics{fj_04b_data_pot5_panel_2.pdf}
        }

        \vspace{-6pt}

        % \figcaption{ 
        \caption{
            Nameraná prevodová charakteristika pre prevádzkovú podmienku $5$ [V]. 
        }
        \label{fj_04b_data_pot5_panel_2}
    }%vbox

\end{figure}
% \end{center}



\subsection{Aproximácia s využitím viacfaktorového modelu}

Viac faktorovým modelom rozumieme taký, ktorý ma viac ako jeden vstup. V tomto prípade môžeme signál o polohe potenciometra využiť ako ďalší vstup do modelu. 

% \begin{center}
    \begin{figure}[!h]

    \vspace{-10pt}

    \vbox{%
        \makebox[\textwidth][c]{%
        \includegraphics{fj_06__panel_2.png}
        }

        \vspace{-12pt}

        % \figcaption{ 
        \caption{
            Červenou farbou sú znázornené tie hodnoty v ustálonom stave, pre ktoré platí, že výstupný signál má hodnotu vyššiu ako $9,9$ [V]. Modrou farbou tie, pre ktoré platí, že výstupný signál má hodnotu nižšiu ako $0,1$ [V]. Zelenou farbou ostatné.  
        }
        \label{fj_06__panel_2}
    }%vbox

\end{figure}
% \end{center}

Na obrázku \ref{fj_06__panel_2} sú hodnoty v ustálenom stave znázornené ako funkcia dvoch premenných. Funkčnou hodnotou je výstupný signál a argumentmi sú vstupný signál a~prevádzková podmienka. Farebne sú odlíšené prípady, keď výstupný signál je saturovaný a keď nie je. Ak si odmyslíme saturované oblasti, potom môžeme usúdiť, že hodnoty v ustálenom stave tvoria rovinu v trojrozmernom priestore.

Modelom nesaturovanej oblasti nech je funkčná závislosť v tvare
\begin{equation}
    \hat y = f_4(u_1, u_2) = \Theta_{11}\ u_1 + \Theta_{12}\ u_2 + \Theta_0
\end{equation}
kde $u_1$ je vstupný signál, $u_2$ je prevádzková podmienka a $\Theta_{11}$, $\Theta_{12}$ a $\Theta_0$ sú parametre, ktoré je potrebné určiť.

Pre tento prípad boli v zmysle metódy najmenších štvorcov nájdené hodnoty parametrov $\Theta_{11} = 1,167 $, $\Theta_{12} = -0,37876$ a $\Theta_0 = 0,81$. Grafické porovnanie takéhoto modelu s nameranými hodnotami v nesaturovanej oblasti je znázornené na obrázku \ref{fj_07__panel_1}.

% \begin{center}
    \begin{figure}[!h]

    \vspace{-10pt}

    \vbox{%
        \makebox[\textwidth][c]{%
        \includegraphics{fj_07__panel_1.png}
        }

        \vspace{-12pt}

        % \figcaption{ 
        \caption{
%  
        }
        \label{fj_07__panel_1}
    }%vbox

\end{figure}
% \end{center}































% \newpage
\clearpage
% \pagebreak

\section{Príloha - vizualizácia získaných dát}


\vspace{-6pt}

\paragraph{Meranie ustálených hodnôt pri prevádzkovej podmienke 0 [V]}

Pozri obrázok \ref{fj_01_data_pot0}.

\vspace{-6pt}


\paragraph{Meranie ustálených hodnôt pri prevádzkovej podmienke 1 [V]}

\begin{center}

    \vbox{%
        \makebox[\textwidth][c]{%
        \includegraphics{fj_01_data_pot1_panel_1.pdf}
        }

        \makebox[\textwidth][c]{%
        \includegraphics{fj_01_data_pot1_panel_3.pdf}
        }

        \makebox[\textwidth][c]{%
        \includegraphics{fj_01_data_pot1_panel_2.pdf}
        }

        \figcaption{ 
            Získané dáta pre vstupný signál a výstupný signál systému LMOT pri manuálne nastavenej prevádzkovej podmienke $1$ [V].
        }
        \label{fj_01_data_pot1}
    }%vbox

\end{center}

% \vfill

\vspace{-6pt}

\paragraph{Meranie ustálených hodnôt pri prevádkovej podmienke 2 [V]}

\begin{center}

    \vbox{%
        \makebox[\textwidth][c]{%
        \includegraphics{fj_01_data_pot2_panel_1.pdf}
        }

        \makebox[\textwidth][c]{%
        \includegraphics{fj_01_data_pot2_panel_3.pdf}
        }

        \makebox[\textwidth][c]{%
        \includegraphics{fj_01_data_pot2_panel_2.pdf}
        }

        \figcaption{ 
            Získané dáta pre vstupný signál a výstupný signál systému LMOT pri manuálne nastavenej prevádzkovej podmienke $2$ [V].
        }
        \label{fj_01_data_pot2}
    }%vbox

\end{center}

% \vfill

% \phantom{}

\pagebreak



\paragraph{Meranie ustálených hodnôt pri prevádzkovej podmienke 3 [V]}

\begin{center}

    \vbox{%
        \makebox[\textwidth][c]{%
        \includegraphics{fj_01_data_pot3_panel_1.pdf}
        }

        \makebox[\textwidth][c]{%
        \includegraphics{fj_01_data_pot3_panel_3.pdf}
        }

        \makebox[\textwidth][c]{%
        \includegraphics{fj_01_data_pot3_panel_2.pdf}
        }

        \figcaption{ 
            Získané dáta pre vstupný signál a výstupný signál systému LMOT pri manuálne nastavenej prevádzkovej podmienke $3$ [V].
        }
        \label{fj_01_data_pot3}
    }%vbox

\end{center}

\vfill

\paragraph{Meranie ustálených hodnôt pri prevádzkovej podmienke 4 [V]}

\begin{center}

    \vbox{%
        \makebox[\textwidth][c]{%
        \includegraphics{fj_01_data_pot4_panel_1.pdf}
        }

        \makebox[\textwidth][c]{%
        \includegraphics{fj_01_data_pot4_panel_3.pdf}
        }

        \makebox[\textwidth][c]{%
        \includegraphics{fj_01_data_pot4_panel_2.pdf}
        }

        \figcaption{ 
            Získané dáta pre vstupný signál a výstupný signál systému LMOT pri manuálne nastavenej prevádzkovej podmienke $4$ [V].
        }

        \label{fj_01_data_pot4}
    }%vbox

\end{center}

% \vfill

% \phantom{}

\pagebreak

\paragraph{Meranie ustálených hodnôt pri prevádzkovej podmienke 5 [V]}

\begin{center}

    \vbox{%
        \makebox[\textwidth][c]{%
        \includegraphics{fj_01_data_pot5_panel_1.pdf}
        }

        \makebox[\textwidth][c]{%
        \includegraphics{fj_01_data_pot5_panel_3.pdf}
        }

        \makebox[\textwidth][c]{%
        \includegraphics{fj_01_data_pot5_panel_2.pdf}
        }

        \figcaption{ 
            Získané dáta pre vstupný signál a výstupný signál systému LMOT pri manuálne nastavenej prevádzkovej podmienke $5$ [V].
        }

        \label{fj_01_data_pot5}

    }%vbox

\end{center}

\vfill

\paragraph{Meranie ustálených hodnôt pri prevádzkovej podmienke 6 [V]}

\begin{center}

    \vbox{%
        \makebox[\textwidth][c]{%
        \includegraphics{fj_01_data_pot6_panel_1.pdf}
        }

        \makebox[\textwidth][c]{%
        \includegraphics{fj_01_data_pot6_panel_3.pdf}
        }

        \makebox[\textwidth][c]{%
        \includegraphics{fj_01_data_pot6_panel_2.pdf}
        }

        \figcaption{ 
            Získané dáta pre vstupný signál a výstupný signál systému LMOT pri manuálne nastavenej prevádzkovej podmienke $6$ [V].
        }

        \label{fj_01_data_pot6}

    }%vbox

\end{center}

% \vfill

% \phantom{}

\pagebreak

\paragraph{Meranie ustálených hodnôt pri prevádzkovej podmienke 7 [V]}

\begin{center}

    \vbox{%
        \makebox[\textwidth][c]{%
        \includegraphics{fj_01_data_pot7_panel_1.pdf}
        }

        \makebox[\textwidth][c]{%
        \includegraphics{fj_01_data_pot7_panel_3.pdf}
        }

        \makebox[\textwidth][c]{%
        \includegraphics{fj_01_data_pot7_panel_2.pdf}
        }

        \figcaption{ 
            Získané dáta pre vstupný signál a výstupný signál systému LMOT pri manuálne nastavenej prevádzkovej podmienke $7$ [V].
        }

        \label{fj_01_data_pot7}

    }%vbox

\end{center}

\vfill

\paragraph{Meranie ustálených hodnôt pri prevádzkovej podmienke 8 [V]}

\begin{center}

    \vbox{%
        \makebox[\textwidth][c]{%
        \includegraphics{fj_01_data_pot8_panel_1.pdf}
        }

        \makebox[\textwidth][c]{%
        \includegraphics{fj_01_data_pot8_panel_3.pdf}
        }

        \makebox[\textwidth][c]{%
        \includegraphics{fj_01_data_pot8_panel_2.pdf}
        }

        \figcaption{ 
            Získané dáta pre vstupný signál a výstupný signál systému LMOT pri manuálne nastavenej prevádzkovej podmienke $8$ [V].
        }

        \label{fj_01_data_pot8}

    }%vbox

\end{center}

% \vfill

% \phantom{}

\pagebreak

\paragraph{Meranie ustálených hodnôt pri prevádzkovej podmienke 9 [V]}

\begin{center}

    \vbox{%
        \makebox[\textwidth][c]{%
        \includegraphics{fj_01_data_pot9_panel_1.pdf}
        }

        \makebox[\textwidth][c]{%
        \includegraphics{fj_01_data_pot9_panel_3.pdf}
        }

        \makebox[\textwidth][c]{%
        \includegraphics{fj_01_data_pot9_panel_2.pdf}
        }

        \figcaption{ 
            Získané dáta pre vstupný signál a výstupný signál systému LMOT pri manuálne nastavenej prevádzkovej podmienke $9$ [V].
        }

        \label{fj_01_data_pot9}

    }%vbox

\end{center}

\vfill

\paragraph{Meranie ustálených hodnôt pri prevádzkovej podmienke 10 [V]}

\begin{center}

    \vbox{%
        \makebox[\textwidth][c]{%
        \includegraphics{fj_01_data_pot10_panel_1.pdf}
        }

        \makebox[\textwidth][c]{%
        \includegraphics{fj_01_data_pot10_panel_3.pdf}
        }

        \makebox[\textwidth][c]{%
        \includegraphics{fj_01_data_pot10_panel_2.pdf}
        }

        \figcaption{ 
            Získané dáta pre vstupný signál a výstupný signál systému LMOT pri manuálne nastavenej prevádzkovej podmienke $10$ [V].
        }

        \label{fj_01_data_pot10}

    }%vbox

\end{center}

% \vfill

% \phantom{}

















% -----------------------------------------------------------------------------

\end{document}

% -----------------------------------------------------------------------------