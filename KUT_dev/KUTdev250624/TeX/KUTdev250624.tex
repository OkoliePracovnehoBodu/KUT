\documentclass[a4paper, 10pt, ]{article}

\input{./COMMONFILES/preamble.tex}

% -----------------------------------------------------------------------------

\def\oznacenieCelku{Kolekcia učebných textov}

% -----------------------------------------------------------------------------


\def\KUTporadoveCislo{dev250624}

% \def\oznacenieVerzie{v0.9}
\def\oznacenieVerzie{\phantom{v1.0}}

\def\mesiacRok{jún 2025}

\def\authorslabel{MT}






% -----------------------------------------------------------------------------

\begin{document}

% -----------------------------------------------------------------------------
% Uvodny nadpis

\noindent
\parbox[t][18mm][c]{0.3\textwidth}{%
\raisebox{-0.9\height}{%
\phantom{.}\includegraphics[height=18mm]{./COMMONFILES/URKFEIlogo.pdf}%
}%
}%
\parbox[t][18mm][c]{0.7\textwidth}{%
\raggedleft

\sffamily
\fontsize{16pt}{18pt}
\fontseries{sbc}
\selectfont

\noindent
\textcolor[rgb]{0.75, 0.75, 0.75}{\textls[25]{\oznacenieCelku}}
}%

\noindent
\parbox[t][16mm][b]{0.5\textwidth}{%
\raggedright

\color{Gray}
\sffamily

\fontsize{12pt}{12pt}
\selectfont
\mesiacRok

\fontsize{6pt}{10pt}
\selectfont
github.com/OkoliePracovnehoBodu/KUT

\fontsize{8pt}{10pt}
\selectfont
\authorslabel




}%
\parbox[t][16mm][b]{0.5\textwidth}{%
\raggedleft

\sffamily

\fontsize{6pt}{6pt}
\selectfont

\textcolor[rgb]{0.68, 0.68, 0.68}{\oznacenieVerzie}


\fontsize{14pt}{14pt}
\selectfont

\bfseries

\includegraphics[height=12pt]{./COMMONFILES/KUT_logo_v0.1.pdf}%
{%
\textls[-50]{\KUTporadoveCislo}
}%
}%

% -----------------------------------------------------------------------------




\vspace{6mm}

% ---------------------------------------------
\sffamily
\bfseries
\fontsize{18pt}{21pt}
\selectfont

\begin{flushleft}
    Náučný príklad: meranie prevodovej charakteristiky dynamického systému LMOT
\end{flushleft}

\bigskip

% -----------------------------------------------------------------------------
\normalsize
\normalfont
% -----------------------------------------------------------------------------

\lstset{style=mystyle}










\noindent
\lettrine[lines=1, nindent=1pt, loversize=0.0]{C}{ieľom} 
textu je opis laboratórneho dynamického systému LMOT z hľadiska jeho statických vlastností.


\section{Opis dynamického systému LMOT}

TODO\ldots


\section{Návrh merania}

Z opisu predmetného dynamického systému vyplýva, že systém má jeden výstupný signál, jeden vstupný signál a manuálne nastaviteľnú prevádzkovú podmienku.

\paragraph{Rozsahy a jednotky signálov}

Vstupný a výstupný signál nadobúdajú hodnoty v~rozsahu $0$ až $10$ pričom ide o~napäťové signály vo voltoch [V].

Prevádzková podmienka systému sa nastavuje manuálne otáčaním potenciometra. Signál o polohe potenciometra nadobúda hodnoty v rozsahu $0$ [V] až $10$ [V].

\paragraph{Voľba ustálených hodnôt vstupov}

O predmetnom systéme je známe, že výstup systému sa ustáli vždy ak sú vstupy systému ustálené. Pre vyšetrovanie ustálených stavov je teda možné využiť celý rozsah vstupného signálu a celý rozsah prevádzkových podmienok.

Návrh uvažuje ustálené hodnoty vstupného signálu uvedené v tabuľke~\ref{tab:ustalene_hodnoty_vstupneho_signalu} a zároveň ustálené hodnoty reprezentujúce prevádzkové podmienky podľa tabuľky \ref{tab:ustalene_hodnoty_podmienok}.


\begin{center}

\vspace{-10pt}    
    
\tabcaption{Ustálené hodnoty vstupného signálu [V]}
\label{tab:ustalene_hodnoty_vstupneho_signalu}

\lstyle

\begin{tabular*}{\textwidth}{@{ \extracolsep{\fill}} ccccccccccc}
\toprule
0 & 1 & 2 & 3 & 4 & 5 & 6 & 7 & 8 & 9 & 10 \\
\bottomrule
\end{tabular*}


\tabcaption{Ustálené hodnoty signálu o prevádkových podmienkach [V]}
\label{tab:ustalene_hodnoty_podmienok}

\lstyle

\begin{tabular*}{\textwidth}{@{ \extracolsep{\fill}} ccccccccccc}
\toprule
0 & 1 & 2 & 3 & 4 & 5 & 6 & 7 & 8 & 9 & 10 \\
\bottomrule
\end{tabular*}

\end{center}


\paragraph{Voľba časového intervalu pre ustálenie výstupu systému}

Empirické skúsenosti so systémom ukazujú, že z praktického hľadiska sa systém ustáli do $15$ sekúnd po zmene na vstupe systému. Ukazuje sa však aj náchylnosť systému k poruchám spôsobeným zväčša mechanickými nedostatkami a vibráciami zrejme spôsobujúcimi zmeny trenia v mechanických častiach systému. Pre pozorovanie a~vyhodnotenie vplyvu týchto porúch v ustálenom stave je časový interval pre ustálenie zvolený na $120$ sekúnd.



\paragraph{Postup merania}

Vzhľadom na uvedené voľby ustálených hodnôt a časového intervalu návrh predpokladá nasledovný postup.

\begin{enumerate}[leftmargin=0pt, labelsep=4mm, itemsep=0pt]
    \item Manuálne nastavenie prevádzkových podmienok na hodnotu z tabuľky~\ref{tab:ustalene_hodnoty_podmienok}.
    \item Postupná zmena vstupného signálu na hodnoty z tabuľky~\ref{tab:ustalene_hodnoty_vstupneho_signalu} so zvoleným časovým intervalom. Takúto postupnú zmenu vyjadruje nasledujúca tabuľka \ref{tab:postupna_zmena_vstupneho_signalu}.
    
\end{enumerate}


\begin{center}

\vspace{-10pt}    

\tabcaption{Postupná zmena vstupného signálu}
\label{tab:postupna_zmena_vstupneho_signalu}

\lstyle

\begin{tabular*}{\textwidth}{@{ \extracolsep{\fill}} cc}
\toprule
Čas zmeny vstupného signálu [s] & Hodnota vstupného signál [V] \\
\midrule
0 & 0 \\
120 & 1 \\
240 & 2 \\
360 & 3 \\
480 & 4 \\
600 & 5 \\
720 & 6 \\
840 & 7 \\
960 & 8 \\
1080 & 9 \\
1200 & 10 \\
\bottomrule
\end{tabular*}

\end{center}

Celková dĺžka merania je teda $1200 + 120 = 1320$ sekúnd a počas tejto doby sú prevádzkové podmienky konštantné.



\section{Získané dáta}

Príklad získaných dát podľa uvedeného postupu je znázornený na nasledujúcom obrázku.


\begin{center}

    \vbox{%
        \makebox[\textwidth][c]{%
        \includegraphics{fj_01_data_pot0_panel_1.pdf}
        }

        \makebox[\textwidth][c]{%
        \includegraphics{fj_01_data_pot0_panel_3.pdf}
        }

        \makebox[\textwidth][c]{%
        \includegraphics{fj_01_data_pot0_panel_2.pdf}
        }

        \figcaption{ 
            Získané dáta pre vstupný signál a výstupný signál systému LMOT pri manuálne nastavenej prevádzkovej podmienke $0$ [V].
        }
        \label{fj_01_data_pot0}
    }%vbox

\end{center}

Získané dáta pre ostatné prevádzkové podmienky sú uvedené v~prílohách tohto textu. Vizualizované sú tu tak dáta pre všetky prevádzkové podmienky, ktoré sú v~tabuľke~\ref{tab:ustalene_hodnoty_podmienok}.





\section{Spracovanie dát}








\begin{center}

    \vbox{%
        \makebox[\textwidth][c]{%
        \includegraphics{fj_02_data_pot0_panel_1.pdf}
        }

        \makebox[\textwidth][c]{%
        \includegraphics{fj_02_data_pot0_panel_3.pdf}
        }


        \figcaption{ 
            Získané dáta pre vstupný signál a výstupný signál systému LMOT pri manuálne nastavenej prevádzkovej podmienke $0$ [V].
        }
        \label{fj_02_data_pot0}
    }%vbox

\end{center}

















\newpage

\section{Prílohy}


\paragraph{Meranie ustálených hodnôt pri prevádkovej podmienke 1 [V]}

\begin{center}

    \vbox{%
        \makebox[\textwidth][c]{%
        \includegraphics{fj_01_data_pot1_panel_1.pdf}
        }

        \makebox[\textwidth][c]{%
        \includegraphics{fj_01_data_pot1_panel_3.pdf}
        }

        \makebox[\textwidth][c]{%
        \includegraphics{fj_01_data_pot1_panel_2.pdf}
        }

        \figcaption{ 
            Získané dáta pre vstupný signál a výstupný signál systému LMOT pri manuálne nastavenej prevádzkovej podmienke $1$ [V].
        }
        \label{fj_01_data_pot1}
    }%vbox

\end{center}

\vfill

\paragraph{Meranie ustálených hodnôt pri prevádkovej podmienke 2 [V]}

\begin{center}

    \vbox{%
        \makebox[\textwidth][c]{%
        \includegraphics{fj_01_data_pot2_panel_1.pdf}
        }

        \makebox[\textwidth][c]{%
        \includegraphics{fj_01_data_pot2_panel_3.pdf}
        }

        \makebox[\textwidth][c]{%
        \includegraphics{fj_01_data_pot2_panel_2.pdf}
        }

        \figcaption{ 
            Získané dáta pre vstupný signál a výstupný signál systému LMOT pri manuálne nastavenej prevádzkovej podmienke $2$ [V].
        }
        \label{fj_01_data_pot2}
    }%vbox

\end{center}

\vfill

\phantom{}

\pagebreak



\paragraph{Meranie ustálených hodnôt pri prevádkovej podmienke 3 [V]}

\begin{center}

    \vbox{%
        \makebox[\textwidth][c]{%
        \includegraphics{fj_01_data_pot3_panel_1.pdf}
        }

        \makebox[\textwidth][c]{%
        \includegraphics{fj_01_data_pot3_panel_3.pdf}
        }

        \makebox[\textwidth][c]{%
        \includegraphics{fj_01_data_pot3_panel_2.pdf}
        }

        \figcaption{ 
            Získané dáta pre vstupný signál a výstupný signál systému LMOT pri manuálne nastavenej prevádzkovej podmienke $3$ [V].
        }
        \label{fj_01_data_pot3}
    }%vbox

\end{center}

\vfill

\paragraph{Meranie ustálených hodnôt pri prevádkovej podmienke 4 [V]}

\begin{center}

    \vbox{%
        \makebox[\textwidth][c]{%
        \includegraphics{fj_01_data_pot4_panel_1.pdf}
        }

        \makebox[\textwidth][c]{%
        \includegraphics{fj_01_data_pot4_panel_3.pdf}
        }

        \makebox[\textwidth][c]{%
        \includegraphics{fj_01_data_pot4_panel_2.pdf}
        }

        \figcaption{ 
            Získané dáta pre vstupný signál a výstupný signál systému LMOT pri manuálne nastavenej prevádzkovej podmienke $4$ [V].
        }

        \label{fj_01_data_pot4}
    }%vbox

\end{center}

\vfill

\phantom{}

\pagebreak

\paragraph{Meranie ustálených hodnôt pri prevádkovej podmienke 5 [V]}

\begin{center}

    \vbox{%
        \makebox[\textwidth][c]{%
        \includegraphics{fj_01_data_pot5_panel_1.pdf}
        }

        \makebox[\textwidth][c]{%
        \includegraphics{fj_01_data_pot5_panel_3.pdf}
        }

        \makebox[\textwidth][c]{%
        \includegraphics{fj_01_data_pot5_panel_2.pdf}
        }

        \figcaption{ 
            Získané dáta pre vstupný signál a výstupný signál systému LMOT pri manuálne nastavenej prevádzkovej podmienke $5$ [V].
        }

        \label{fj_01_data_pot5}

    }%vbox

\end{center}

\vfill

\paragraph{Meranie ustálených hodnôt pri prevádkovej podmienke 6 [V]}

\begin{center}

    \vbox{%
        \makebox[\textwidth][c]{%
        \includegraphics{fj_01_data_pot6_panel_1.pdf}
        }

        \makebox[\textwidth][c]{%
        \includegraphics{fj_01_data_pot6_panel_3.pdf}
        }

        \makebox[\textwidth][c]{%
        \includegraphics{fj_01_data_pot6_panel_2.pdf}
        }

        \figcaption{ 
            Získané dáta pre vstupný signál a výstupný signál systému LMOT pri manuálne nastavenej prevádzkovej podmienke $6$ [V].
        }

        \label{fj_01_data_pot6}

    }%vbox

\end{center}

\vfill

\phantom{}

\pagebreak

\paragraph{Meranie ustálených hodnôt pri prevádkovej podmienke 7 [V]}

\begin{center}

    \vbox{%
        \makebox[\textwidth][c]{%
        \includegraphics{fj_01_data_pot7_panel_1.pdf}
        }

        \makebox[\textwidth][c]{%
        \includegraphics{fj_01_data_pot7_panel_3.pdf}
        }

        \makebox[\textwidth][c]{%
        \includegraphics{fj_01_data_pot7_panel_2.pdf}
        }

        \figcaption{ 
            Získané dáta pre vstupný signál a výstupný signál systému LMOT pri manuálne nastavenej prevádzkovej podmienke $7$ [V].
        }

        \label{fj_01_data_pot7}

    }%vbox

\end{center}

\vfill

\paragraph{Meranie ustálených hodnôt pri prevádkovej podmienke 8 [V]}

\begin{center}

    \vbox{%
        \makebox[\textwidth][c]{%
        \includegraphics{fj_01_data_pot8_panel_1.pdf}
        }

        \makebox[\textwidth][c]{%
        \includegraphics{fj_01_data_pot8_panel_3.pdf}
        }

        \makebox[\textwidth][c]{%
        \includegraphics{fj_01_data_pot8_panel_2.pdf}
        }

        \figcaption{ 
            Získané dáta pre vstupný signál a výstupný signál systému LMOT pri manuálne nastavenej prevádzkovej podmienke $8$ [V].
        }

        \label{fj_01_data_pot8}

    }%vbox

\end{center}

\vfill

\phantom{}

\pagebreak

\paragraph{Meranie ustálených hodnôt pri prevádkovej podmienke 9 [V]}

\begin{center}

    \vbox{%
        \makebox[\textwidth][c]{%
        \includegraphics{fj_01_data_pot9_panel_1.pdf}
        }

        \makebox[\textwidth][c]{%
        \includegraphics{fj_01_data_pot9_panel_3.pdf}
        }

        \makebox[\textwidth][c]{%
        \includegraphics{fj_01_data_pot9_panel_2.pdf}
        }

        \figcaption{ 
            Získané dáta pre vstupný signál a výstupný signál systému LMOT pri manuálne nastavenej prevádzkovej podmienke $9$ [V].
        }

        \label{fj_01_data_pot9}

    }%vbox

\end{center}

\vfill

\paragraph{Meranie ustálených hodnôt pri prevádkovej podmienke 10 [V]}

\begin{center}

    \vbox{%
        \makebox[\textwidth][c]{%
        \includegraphics{fj_01_data_pot10_panel_1.pdf}
        }

        \makebox[\textwidth][c]{%
        \includegraphics{fj_01_data_pot10_panel_3.pdf}
        }

        \makebox[\textwidth][c]{%
        \includegraphics{fj_01_data_pot10_panel_2.pdf}
        }

        \figcaption{ 
            Získané dáta pre vstupný signál a výstupný signál systému LMOT pri manuálne nastavenej prevádzkovej podmienke $10$ [V].
        }

        \label{fj_01_data_pot10}

    }%vbox

\end{center}

\vfill

\phantom{}




















% -----------------------------------------------------------------------------

\end{document}

% -----------------------------------------------------------------------------