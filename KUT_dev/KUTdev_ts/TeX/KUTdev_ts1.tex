\documentclass[a4paper, 10pt, ]{article}

\input{./COMMONFILES/preamble.tex}

% -----------------------------------------------------------------------------

\def\oznacenieCelku{Kolekcia učebných textov}

% -----------------------------------------------------------------------------


\def\KUTporadoveCislo{devTS1}

% \def\oznacenieVerzie{v0.9}
\def\oznacenieVerzie{\phantom{v1.0}}

\def\mesiacRok{september 2025}

\def\authorslabel{AS}






% -----------------------------------------------------------------------------

\begin{document}

% -----------------------------------------------------------------------------
% Uvodny nadpis

\noindent
\parbox[t][18mm][c]{0.3\textwidth}{%
\raisebox{-0.9\height}{%
\phantom{.}\includegraphics[height=18mm]{./COMMONFILES/URKFEIlogo.pdf}%
}%
}%
\parbox[t][18mm][c]{0.7\textwidth}{%
\raggedleft

\sffamily
\fontsize{16pt}{18pt}
\fontseries{sbc}
\selectfont

\noindent
\textcolor[rgb]{0.75, 0.75, 0.75}{\textls[25]{\oznacenieCelku}}
}%

\noindent
\parbox[t][16mm][b]{0.5\textwidth}{%
\raggedright

\color{Gray}
\sffamily

\fontsize{12pt}{12pt}
\selectfont
\mesiacRok

\fontsize{6pt}{10pt}
\selectfont
github.com/OkoliePracovnehoBodu/KUT

\fontsize{8pt}{10pt}
\selectfont
\authorslabel




}%
\parbox[t][16mm][b]{0.5\textwidth}{%
\raggedleft

\sffamily

\fontsize{6pt}{6pt}
\selectfont

\textcolor[rgb]{0.68, 0.68, 0.68}{\oznacenieVerzie}


\fontsize{14pt}{14pt}
\selectfont

\bfseries

\includegraphics[height=12pt]{./COMMONFILES/KUT_logo_v0.1.pdf}%
{%
\textls[-50]{\KUTporadoveCislo}
}%
}%

% -----------------------------------------------------------------------------




\vspace{6mm}

% ---------------------------------------------
\sffamily
\bfseries
\fontsize{18pt}{21pt}
\selectfont

\begin{flushleft}
    Statické vlastnosti systému: TS
\end{flushleft}

\bigskip

% -----------------------------------------------------------------------------
\normalsize
\normalfont
% -----------------------------------------------------------------------------

\lstset{style=mystyle}


\noindent
\lettrine[lines=1, nindent=1pt, loversize=0.0]{C}{ieľom} 
textu je opis statických vlastností modelu technického systému (TS).

% -----------------Section1--------------------------------------------

\section{Statické vlastnosti modelu TS}

\noindent
Na účely demonštrácie statických vlastností bol experimentálne 
zvolený systém s dvoma vstupmi: výkon ohrevu špirály a 
výkon ventilátora. Ide teda o MIMO systém 
(Multi-Input Multi-Output), pričom v tejto časti sa zameriavame výhradne 
na správanie vstupných signálov. Výstupné veličiny budú analyzované 
v nasledujúcich kapitolách. 

\medskip

\noindent
Prevodová charakteristika predstavuje závislosť medzi ustálenými 
hodnotami vstupu a zodpovedajúcimi ustálenými hodnotami výstupu. 
V tejto fáze je preto potrebné ukázať priebeh samotných vstupov, 
ktoré budú slúžiť ako základ pre ďalšiu analýzu. 

\medskip

\noindent
Výkon ohrevu špirály bol menený v krokoch po~1, pričom každá zmena 
nastala v intervale 100~sekúnd. Rozsah hodnôt bol od 0 do 10. 

\begin{center}
    \vbox{%
        \makebox[\textwidth][c]{%
        \includegraphics[width=1.5\textwidth]{prev_spir.png}
        }
        \figcaption{ 
            Priebeh vstupu: výkon ohrevu špirály počas experimentu.
        }
        \label{fig_spirala_time}
    }%vbox
\end{center}

\noindent
Experiment bol zopakovaný trikrát, pričom vždy bola zvolená konštantná 
hodnota druhého vstupu – výkonu ventilátora. Tieto hodnoty boli 
nastavené na 3, 6 a 9. 

\medskip

\newpage

\noindent
Na nasledujúcich obrázkoch sú zobrazené priebehy vstupu pre jednotlivé 
experimenty:

\begin{center}
    \vbox{%
        \makebox[\textwidth][c]{%
        \includegraphics[width=1.5\textwidth]{prev3_vent.png}
        }
        \figcaption{ 
            Priebeh vstupu: výkon ventilátora pre experiment 1.
        }
        \label{fig_exp1}
    }%vbox
\end{center}

\begin{center}
    \vbox{%
        \makebox[\textwidth][c]{%
        \includegraphics[width=1.5\textwidth]{prev6_vent.png}
        }
        \figcaption{ 
            Priebeh vstupu: výkon ventilátora pre experiment 2.
        }
        \label{fig_exp2}
    }%vbox
\end{center}

\begin{center}
    \vbox{%
        \makebox[\textwidth][c]{%
        \includegraphics[width=1.5\textwidth]{prev9_vent.png}
        }
        \figcaption{ 
            Priebeh vstupu: výkon ventilátora pre experiment 3.
        }
        \label{fig_exp3}
    }%vbox
\end{center}

% -------Section2--------------------------------------------

\newpage
\section{Výsledky experimentu 1}

\noindent
V tejto časti sú prezentované výsledky prvého experimentu. 
Cieľom bolo zaznamenať reakciu systému na postupné skokové zmeny vstupov. 
Pre každý vstupný krok bola reakcia systému meraná dvomi senzormi. 

\medskip

\noindent
Aby sme získali ustálené hodnoty výstupov, 
pre každý skok sme vypočítali priemernú hodnotu zo všetkých meraní 
v posledných 10 sekundách daného intervalu. 
Tieto priemery sú v ďalšej analýze považované za ustálené hodnoty výstupov. 

\subsection*{Výstupné signály systému}

\begin{center}
    \vbox{%
        \makebox[\textwidth][c]{%
        \includegraphics[width=1.5\textwidth]{prev3_snim1.png}
        }
        \figcaption{ 
            Výstupný signál zo senzora 1.
        }
        \label{fig_exp1_sen1}
    }%vbox
\end{center}

\begin{center}
    \vbox{%
        \makebox[\textwidth][c]{%
        \includegraphics[width=1.5\textwidth]{prev3_snim2.png}
        }
        \figcaption{ 
            Výstupný signál zo senzora 2.
        }
        \label{fig_exp1_sen2}
    }%vbox
\end{center}

\subsection*{Prevodové charakteristiky}

\begin{center}
    \vbox{%
        \makebox[\textwidth][c]{%
        \includegraphics[width=1.5\textwidth]{prev3_stat1.png}
        }
        \figcaption{ 
            Prevodová charakteristika pre senzor 1 – závislosť ustálenej hodnoty výstupu od vstupu.
        }
        \label{fig_exp1_trans1}
    }%vbox
\end{center}

\begin{center}
    \vbox{%
        \makebox[\textwidth][c]{%
        \includegraphics[width=1.5\textwidth]{prev3_stat2.png}
        }
        \figcaption{ 
            Prevodová charakteristika pre senzor 2 – závislosť ustálenej hodnoty výstupu od vstupu.
        }
        \label{fig_exp1_trans2}
    }%vbox
\end{center}

%---------------Section3--------------------------------------------

\newpage
\section{Výsledky experimentu 2}

\noindent
Výstupné signály a prevodové charakteristiky zaznamenané počas experimentu 2.

\subsection*{Výstupné signály systému}

\begin{center}
    \vbox{%
        \makebox[\textwidth][c]{%
        \includegraphics[width=1.5\textwidth]{prev6_snim1.png}
        }
        \figcaption{ 
            Výstupný signál zo senzora 1.
        }
        \label{fig_exp1_sen1}
    }%vbox
\end{center}

\begin{center}
    \vbox{%
        \makebox[\textwidth][c]{%
        \includegraphics[width=1.5\textwidth]{prev6_snim2.png}
        }
        \figcaption{ 
            Výstupný signál zo senzora 2.
        }
        \label{fig_exp1_sen2}
    }%vbox
\end{center}

\subsection*{Prevodové charakteristiky}

\begin{center}
    \vbox{%
        \makebox[\textwidth][c]{%
        \includegraphics[width=1.5\textwidth]{prev6_stat1.png}
        }
        \figcaption{ 
            Prevodová charakteristika pre senzor 1 – závislosť ustálenej hodnoty výstupu od vstupu.
        }
        \label{fig_exp1_trans1}
    }%vbox
\end{center}

\begin{center}
    \vbox{%
        \makebox[\textwidth][c]{%
        \includegraphics[width=1.5\textwidth]{prev6_stat2.png}
        }
        \figcaption{ 
            Prevodová charakteristika pre senzor 2 – závislosť ustálenej hodnoty výstupu od vstupu.
        }
        \label{fig_exp1_trans2}
    }%vbox
\end{center}

%---------------Section4--------------------------------------------

\newpage
\section{Výsledky experimentu 3}

\noindent
Výstupné signály a prevodové charakteristiky zaznamenané počas experimentu 3.

\subsection*{Výstupné signály systému}

\begin{center}
    \vbox{%
        \makebox[\textwidth][c]{%
        \includegraphics[width=1.5\textwidth]{prev9_snim1.png}
        }
        \figcaption{ 
            Výstupný signál zo senzora 1.
        }
        \label{fig_exp1_sen1}
    }%vbox
\end{center}

\begin{center}
    \vbox{%
        \makebox[\textwidth][c]{%
        \includegraphics[width=1.5\textwidth]{prev9_snim2.png}
        }
        \figcaption{ 
            Výstupný signál zo senzora 2.
        }
        \label{fig_exp1_sen2}
    }%vbox
\end{center}

\subsection*{Prevodové charakteristiky}

\begin{center}
    \vbox{%
        \makebox[\textwidth][c]{%
        \includegraphics[width=1.5\textwidth]{prev9_stat1.png}
        }
        \figcaption{ 
            Prevodová charakteristika pre senzor 1 – závislosť ustálenej hodnoty výstupu od vstupu.
        }
        \label{fig_exp1_trans1}
    }%vbox
\end{center}

\begin{center}
    \vbox{%
        \makebox[\textwidth][c]{%
        \includegraphics[width=1.5\textwidth]{prev9_stat2.png}
        }
        \figcaption{ 
            Prevodová charakteristika pre senzor 2 – závislosť ustálenej hodnoty výstupu od vstupu.
        }
        \label{fig_exp1_trans2}
    }%vbox
\end{center}

% -----------------------------------------------------------------------------

\end{document}

% -----------------------------------------------------------------------------