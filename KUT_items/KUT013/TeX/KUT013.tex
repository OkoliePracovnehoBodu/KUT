\documentclass[a4paper, 10pt, ]{article}

\input{./COMMONFILES/preamble.tex}

% -----------------------------------------------------------------------------

\def\oznacenieCelku{Kolekcia učebných textov}

% -----------------------------------------------------------------------------


\def\KUTporadoveCislo{013}

\def\oznacenieVerzie{v0.9}
% \def\oznacenieVerzie{\phantom{v1.0}}

\def\mesiacRok{júl 2025}

\def\authorslabel{MT}






% -----------------------------------------------------------------------------

\begin{document}

% -----------------------------------------------------------------------------
% Uvodny nadpis

\noindent
\parbox[t][18mm][c]{0.3\textwidth}{%
\raisebox{-0.9\height}{%
\phantom{.}\includegraphics[height=18mm]{./COMMONFILES/URKFEIlogo.pdf}%
}%
}%
\parbox[t][18mm][c]{0.7\textwidth}{%
\raggedleft

\sffamily
\fontsize{16pt}{18pt}
\fontseries{sbc}
\selectfont

\noindent
\textcolor[rgb]{0.75, 0.75, 0.75}{\textls[25]{\oznacenieCelku}}
}%

\noindent
\parbox[t][16mm][b]{0.5\textwidth}{%
\raggedright

\color{Gray}
\sffamily

\fontsize{12pt}{12pt}
\selectfont
\mesiacRok

\fontsize{6pt}{10pt}
\selectfont
github.com/OkoliePracovnehoBodu/KUT

\fontsize{8pt}{10pt}
\selectfont
\authorslabel




}%
\parbox[t][16mm][b]{0.5\textwidth}{%
\raggedleft

\sffamily

\fontsize{6pt}{6pt}
\selectfont

\textcolor[rgb]{0.68, 0.68, 0.68}{\oznacenieVerzie}


\fontsize{14pt}{14pt}
\selectfont

\bfseries

\includegraphics[height=12pt]{./COMMONFILES/KUT_logo_v0.1.pdf}%
{%
\textls[-50]{\KUTporadoveCislo}
}%
}%

% -----------------------------------------------------------------------------




\vspace{6mm}

% ---------------------------------------------
\sffamily
\bfseries
\fontsize{18pt}{21pt}
\selectfont

\begin{flushleft}
    Laboratórium kybernetiky:\\ zoznam laboratórnych zariadení
\end{flushleft}

\bigskip

% -----------------------------------------------------------------------------
\normalsize
\normalfont
% -----------------------------------------------------------------------------

\lstset{style=mystyle}










\noindent
\lettrine[lines=1, nindent=1pt, loversize=0.0]{C}{ieľom} 
textu je opis laboratórnych zariadení, výpočtovej techniky a ich rozmiestnenia v Laboratóriu kybernetiky.

\medskip

\noindent
Posledná akzualizácia údajov tu uvedených: júl 2025


\section{Osobné počítače s meracími kartami}

Meriacou kartou sa v tomto prípade rozumie PCI karta rozširujúca možnosti počítača o~funkcie súvisiace s meraním a generovaním analógových a digitálnych signálov.

V Laboratóriu kybernetiky sa používa meracia karta Advantech PCI-1711 a niektoré jej varianty alebo príbuzné karty.



\subsection{Miestnosť D328}

Zoznam počítačov s meracími kartami a ich vybrané parametre sú uvedené v~tabuľke~\ref{tab:PCsWithMeasurementCards}. Rozmiestnenie počítačov s meracími kartami v miestnosti D328 je zobrazené na obrázku \ref{D328map_v1}.





\begin{center}

\tabcaption{PC s meracími kartami a ich vybrané parametre}
\label{tab:PCsWithMeasurementCards}    

\begin{tabular*}{\textwidth}{ @{\extracolsep{\fill}} p{3.5cm} p{8.5cm}}
\toprule
Oznečenie počítača & Vybrané parametre \\
\midrule
\textsf{\textbf{LK\textl{10}}} 
& 
\begin{tabular}{@{}l >{\lstyle}l}
CPU & i3-2100 (rok uvedenia na trh 2011)\\
RAM & 8 GB \\
mer. karta & Advantech PCI-1711
\end{tabular}
\\
\midrule
\textsf{\textbf{LK\textl{23}}} 
& 
\begin{tabular}{@{}l >{\lstyle}l}
CPU & i5-4570 (rok uvedenia na trh 2013)\\
RAM & 8 GB \\
mer. karta & Advantech PCI-1716
\end{tabular}
\\
\midrule
\textsf{\textbf{LK\textl{31}}} 
& 
\begin{tabular}{@{}l >{\lstyle}l}
CPU & i7-4770 (rok uvedenia na trh 2013)\\
RAM & 8 GB \\
mer. karta & Advantech PCI-1711U
\end{tabular}
\\
\midrule
\textsf{\textbf{LK\textl{32}}} 
& 
\begin{tabular}{@{}l >{\lstyle}l}
CPU & i7-2600 (rok uvedenia na trh 2011) \\
RAM & 8 GB \\
mer. karta & Advantech PCI-1711U
\end{tabular}
\\
\bottomrule
\end{tabular*}

\end{center}





\begin{figure}[t]
    \centering

    \makebox[\textwidth][c]{%
    \input{../SVG/D328map_v1.pdf_tex}
    }

    \caption{Rozmiestnenie počítačov s meracími kartami v miestnosti D328} 
    \label{D328map_v1}

\end{figure}



\begin{figure}[t]
    \centering

    \makebox[\textwidth][c]{%
    \input{../SVG/D330map_v1.pdf_tex}
    }

    \caption{Rozmiestnenie počítačov s meracími kartami v miestnosti D330} 
    \label{D330map_v1}

\end{figure}



\subsection{Miestnosť D330}

Zoznam počítačov s meracími kartami a ich vybrané parametre sú uvedené v~tabuľke~\ref{tab:PCsWithMeasurementCards2}. Rozmiestnenie počítačov s meracími kartami v miestnosti D330 je zobrazené na obrázku \ref{D330map_v1}.


\begin{center}

\vbox{%    

\tabcaption{PC s meracími kartami a ich vybrané parametre}
\label{tab:PCsWithMeasurementCards2}    

\begin{tabular*}{\textwidth}{ @{\extracolsep{\fill}} p{3.5cm} p{8.5cm}}
\toprule
Oznečenie počítača & Vybrané parametre \\
\midrule
\textsf{\textbf{LK\textl{11}}} 
& 
\begin{tabular}{@{}l >{\lstyle}l}
CPU & i5-4570 (rok uvedenia na trh 2013)\\
RAM & 8 GB \\
mer. karta & Advantech PCI-1711
\end{tabular}
\\
\midrule
\textsf{\textbf{LK\textl{12}}} 
& 
\begin{tabular}{@{}l >{\lstyle}l}
CPU & i5-4570 (rok uvedenia na trh 2013)\\
RAM & 8 GB \\
mer. karta & Advantech PCI-1711
\end{tabular}
\\
\midrule
\textsf{\textbf{LK\textl{21}}} 
& 
\begin{tabular}{@{}l >{\lstyle}l}
CPU & i5-4570 (rok uvedenia na trh 2013)\\
RAM & 8 GB \\
mer. karta & Advantech PCI-1711
\end{tabular}
\\
\midrule
\textsf{\textbf{LK\textl{22}}} 
& 
\begin{tabular}{@{}l >{\lstyle}l}
CPU & i5-4570 (rok uvedenia na trh 2013)\\
RAM & 8 GB \\
mer. karta & Advantech PCI-1711
\end{tabular}
\\
\bottomrule
\end{tabular*}

}

\end{center}









\section{Laboratórne zariadenia s rozhraním na meraciu kartu}

Ide o laboratórne zariadenia  predstavujúce reálne dynamické systémy, ktoré majú analógové vstupy a výstupy a rozhranie k týmto vstupom a výstupom je dizajnované tak, aby bolo možné ich pripojiť k~meracej karte.

\subsection{Laboratórne zariadenie LMOT}

LMOT je laboratórne zariadenie predstavujúce reálny dynamický systém. Pozostáva z malého jednosmerného motora, tachodynama, ktoré je na spoločnom hriadeli s~motorom, a z elektronických obvodov, ktoré zabezpečujú napájanie motora. Elektronickými obvodmi sú tiež dané dominantné statické a dynamické vlastnosti výsledného systému. Do istej miery je možné tieto vlastnosti meniť manuálnym nastavením príslušného potenciometra.

Systém má jeden vstupný signál a jeden výstupný signál. Výstupný signál je priamo úmerný uhlovej rýchlosti jednosmerného motora, ktorá je snímaná tachodynamom. Vstupný signál ovláda napájanie motora.

Polohou potenciometra je v podstate daná prevádzková podmienka zariadenia. K~dispozícii je signál zodpovedajúci polohe potenciometra a teda tým je k dispozícii informácia o prevádzkovej podmienke systému.

LMOT (čítaj \emph{elmot}) je akronym pre „laboratórny motorček“, prípadne pre „little motor“.




\subsection{Laboratórne zariadenie TS}

TS, skratka od \emph{tepelný systém} (thermal system), je laboratórne zariadenie predstavujúce reálny dynamický systém. Pozostáva zo sklenenej trubice umiestnenej na podstave. Na jednom konci trubice je upevnený ventilátor, ktorý do trubice vháňa vzduch. V trubici hneď za ventilátorom sa nachádza výhrevné teleso (výhrevná špirála). Za špirálou je umiestnený prvý teplotný snímač a druhý je umiestnený na opačnom konci trubice. V~podstave sa nachádza elektronika zabezpečujúca napájanie komponentov zariadenia a~rozhranie k meracej karte.

Dostupnými sú dva vstupné a dva výstupné analógové signály. Prvý vstupný signál ovláda výkon vyhrievacieho telesa. Druhý vstupný signál ovláda výkon ventilátora. Prvý výstupný signál je teplota vzduchu v~trubici hneď za vyhrievacím telesom. Druhý výstupný signál je teplota vzduchu v trubici na opačnom konci od vyhrievacieho telesa.

Z kybernetického hľadiska je zariadenie TS možné prevádzkovať ako mnohovstupový a~mnohovýstupový systém (MIMO systém) alebo ako jednovstupový a jedno výstupový systém (SISO systém). 

Pri SISO systéme je vstupom signál ovládajúci výkon vyhrievacieho telesa. Signál pre ventilátor v podstate určuje prevádzkovú podmienku zariadenia keďže prúdenie vzduchu v trubici vo všeobecnosti vplýva na jeho ohrievanie a teplotu.








\subsection{Miestnosť D328}

Zoznam laboratórnych zariadení s rozhraním na meraciu kartu je uvedený v~tabuľke~\ref{tab:lzD328}. Rozmiestnenie laboratórnych zariadení s rozhraním na meraciu kartu v miestnosti D328 je zobrazené na obrázku \ref{D328map_v2}.




\begin{figure}[t]
    \centering

    \makebox[\textwidth][c]{%
    \input{../SVG/D328map_v2.pdf_tex}
    }

    \figcaption{Rozmiestnenie laboratórnych zariadení v miestnosti D328} 
    \label{D328map_v2}

\end{figure}




\begin{center}

\vbox{%

\tabcaption{}
\label{tab:lzD328}    

\begin{tabular*}{\textwidth}{ @{\extracolsep{\fill}} p{3.5cm} p{8.5cm}<{\raggedright}}
\toprule
Oznečenie zariadenia & Info. \\
\midrule
\textsf{\textbf{LMOT\textl{01}}} 
& 
Pozostáva zo samostatných jednotiek LMOT\textl{01}a a~LMOT\textl{01}b čo umožňuje vytvoriť zložitejší systém s~dvomi vstupmi a dvomi výstupmi (MIMO systém).
\\
\midrule
\textsf{\textbf{TS\textl{01}}} 
& 
-
\\
\midrule
\textsf{\textbf{TS\textl{04}}} 
& 
-
\\
\midrule
\textsf{\textbf{TS\textl{05}}} 
& 
-
\\
\bottomrule
\end{tabular*}

}

\end{center}








\subsection{Miestnosť D330}

Zoznam laboratórnych zariadení s rozhraním na meraciu kartu je uvedený v~tabuľke~\ref{tab:lzD330}. Rozmiestnenie laboratórnych zariadení s rozhraním na meraciu kartu v miestnosti D328 je zobrazené na obrázku \ref{D330map_v2}.






\begin{figure}[t]
    \centering

    \makebox[\textwidth][c]{%
    \input{../SVG/D330map_v2.pdf_tex}
    }

    \figcaption{Rozmiestnenie laboratórnych zariadení v miestnosti D330} 
    \label{D330map_v2}

\end{figure}



\begin{center}

\vbox{%    

\tabcaption{}
\label{tab:lzD330}    

\begin{tabular*}{\textwidth}{ @{\extracolsep{\fill}} p{3.5cm} p{8.5cm}<{\raggedright}}
\toprule
Oznečenie zariadenia & Info. \\
\midrule
\textsf{\textbf{LMOT\textl{02}}} 
& 
Pozostáva zo samostatných jednotiek LMOT\textl{02}a a~LMOT\textl{02}b. Tieto sú prevádzkované samostatne, teda sú pripojené k~dvom rôznym počítačom s meracími kartami.
\\
\midrule
\textsf{\textbf{LMOT\textl{03}}} 
& 
Pozostáva zo samostatných jednotiek LMOT\textl{03}a a~LMOT\textl{03}b. Tieto sú prevádzkované samostatne. LMOT\textl{03}a je pripojené k~počítaču s meracou kartou. LMOT\textl{03}b je pripravené na pripojenie k~počítaču s~meracou kartou, ale aktuálne nie je pripojené.
\\
\bottomrule
\end{tabular*}

}

\end{center}














% -----------------------------------------------------------------------------

\end{document}

% -----------------------------------------------------------------------------