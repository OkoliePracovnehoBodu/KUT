\documentclass[a4paper, 10pt, ]{article}

\input{./COMMONFILES/preamble.tex}

% -----------------------------------------------------------------------------

\def\oznacenieCelku{Kolekcia učebných textov}

% -----------------------------------------------------------------------------


\def\KUTporadoveCislo{016}

% \def\oznacenieVerzie{v0.9}
\def\oznacenieVerzie{\phantom{v1.0}}

\def\mesiacRok{október 2025}

\def\authorslabel{MT}






% -----------------------------------------------------------------------------

\begin{document}

% -----------------------------------------------------------------------------
% Uvodny nadpis

\noindent
\parbox[t][18mm][c]{0.3\textwidth}{%
\raisebox{-0.9\height}{%
\phantom{.}\includegraphics[height=18mm]{./COMMONFILES/URKFEIlogo.pdf}%
}%
}%
\parbox[t][18mm][c]{0.7\textwidth}{%
\raggedleft

\sffamily
\fontsize{16pt}{18pt}
\fontseries{sbc}
\selectfont

\noindent
\textcolor[rgb]{0.75, 0.75, 0.75}{\textls[25]{\oznacenieCelku}}
}%

\noindent
\parbox[t][16mm][b]{0.5\textwidth}{%
\raggedright

\color{Gray}
\sffamily

\fontsize{12pt}{12pt}
\selectfont
\mesiacRok

\fontsize{6pt}{10pt}
\selectfont
\href{https://github.com/OkoliePracovnehoBodu/KUT}{github.com/OkoliePracovnehoBodu/KUT}

\fontsize{8pt}{10pt}
\selectfont
\authorslabel




}%
\parbox[t][16mm][b]{0.5\textwidth}{%
\raggedleft

\sffamily

\fontsize{6pt}{6pt}
\selectfont

\textcolor[rgb]{0.68, 0.68, 0.68}{\oznacenieVerzie}


\fontsize{14pt}{14pt}
\selectfont

\bfseries

\includegraphics[height=12pt]{./COMMONFILES/KUT_logo_v0.1.pdf}%
{%
\textls[-50]{\KUTporadoveCislo}
}%
}%

% -----------------------------------------------------------------------------




\vspace{6mm}

% ---------------------------------------------
\sffamily
\bfseries
\fontsize{18pt}{21pt}
\selectfont

\begin{flushleft}
    Laboratórne zariadenie LMOT:\\ softvér pre začiatok
\end{flushleft}

\bigskip

% -----------------------------------------------------------------------------
\normalsize
\normalfont
% -----------------------------------------------------------------------------

\lstset{style=mystyle}










\noindent
\lettrine[lines=1, nindent=1pt, loversize=0.0]{T}{ext} 
uvádza softvér pre uľahčenie začiatku práce s laboratórnym zariadením LMOT (čítaj \emph{elmot}, akronym pre „laboratórny motorček“, prípadne pre „little motor“). 



\section{Úvodné poznámky}





Nasledujúce informácie nadväzujú na opis laboratórneho zariadenia LMOT a~aktuálneho stavu Laboratória kybernetiky uvedený v~dokumentoch:

\medskip

\noindent
\begin{tabular*}{\textwidth}{ @{} >{\sffamily}p{2.0cm} @{\extracolsep{\fill}} p{11cm}<{\raggedright}}

    \href{run:../../KUT014/TeX/KUT014.pdf}{KUT014} & \href{run:../../KUT014/TeX/KUT014.pdf}{Laboratórne zariadenie LMOT: orientačný prehľad} \\ \addlinespace[3pt]  

    \href{run:../../KUT013/TeX/KUT013.pdf}{KUT013} & \href{run:../../KUT013/TeX/KUT013.pdf}{Laboratórium kybernetiky: zoznam laboratórnych zariadení} \\ \addlinespace[3pt]  

\end{tabular*}

\medskip



\paragraph{MATLAB - Simulink schémy}

Softvérom tu uvedeným sú  MATLAB - Simulink schémy (modely) s nakonfigurovanými blokmi prepájajúcimi prostredie Simulink so zariadením LMOT prostredníctvom meracej karty. 

V laboratóriu je niekoľko zariadení LMOT a~každé je pripojené ku konkrétnej počítačovej zostave. Ku každému zariadeniu prislúcha zodpovedajúca Simulink schéma. Všetky Simulink schémy sú uložené v adresári \texttt{ML} prislúchajúcom k tomuto \textsf{KUT}:

\noindent
% latex-dirtree-gen --ignore TeX,PY,SVG  ./

\dirtree{%
 .1 .
 .2 ML.
 .3 fig.
 .4 LMOT02a\_start.pdf.        
 .4 LMOT02a\_start\_SYSTEM.pdf.
 .3 LMOT01a\_start.slx.        
 .3 LMOT02a\_start.slx.        
 .3 LMOT02b\_start.slx.        
 .3 LMOT03a\_start.slx.        
 .3 LMOT03b\_start.slx.        
}







% \textl{01}

\section{Softvér pre jednotlivé zariadenia}


\subsection{Zariadenie LMOT02a, schéma \texttt{LMOT02a\_start.slx}}

Zariadenie \textsf{LMOT02a} je pripojené k počítačovej zostave \textsf{LK12}. Zodpovedajúca Simulink schéma je \texttt{LMOT02a\_start.slx}. 

Schéma je zobrazená na obr. \ref{LMOT02a_start}. Blok \textl{SYSTEM} predstavuje samotné laboratórne zariadenie. Vstupujúce a vystupujúce signály označené ako \textl{input}, \textl{tacho} a~\textl{potenciometer} zodpovedajú signálom opísaným v dokumente \textsf{KUT014}. 

Vnútorné zapojenie bloku \textl{SYSTEM} pre túto schému je zobrazené na obr. \ref{LMOT02a_start_SYSTEM} a je dané prepojovacimi vodičmi medzi svorkami zariadenia a svorkovnicou pripojenou k~meracej karte. 

\noindent
\vbox{%

    \makebox[\textwidth][c]{%
	\includegraphics[trim=0mm 68mm 0mm 71mm, clip, scale=0.75]{LMOT02a_start.pdf}
	}

    \vspace{-5mm}

	\figcaption{Simulačná schéma}
	\label{LMOT02a_start}

}%vbox




\smallskip



\noindent
\vbox{%

    \makebox[\textwidth][c]{%
	\includegraphics[trim=0mm 67mm 0mm 71mm, clip, scale=0.75]{LMOT02a_start_SYSTEM.pdf}
	}

    \vspace{-5mm}

	\figcaption{Prepojenie signálov schémy \texttt{LMOT02a\_start.slx} na analógové vstupy a výstupy meracej karty.  AO$0$ je analógový výstup meracej karty a~AI$0$, AI$1$ sú analógové vstupy meracej karty.
    }
	\label{LMOT02a_start_SYSTEM}

}%vbox



\subsection{Zariadenie LMOT02b, schéma \texttt{LMOT02b\_start.slx}}

Zariadenie \textsf{LMOT02b} je pripojené k počítačovej zostave \textsf{LK11}. Zodpovedajúca Simulink schéma je \texttt{LMOT02b\_start.slx}. Opis schémy je analogický ako v predchádzajúcej časti.




\subsection{Zariadenie LMOT03a, schéma \texttt{LMOT03a\_start.slx}}

Zariadenie \textsf{LMOT03a} je pripojené k počítačovej zostave \textsf{LK21}. Zodpovedajúca Simulink schéma je \texttt{LMOT03a\_start.slx}. Opis schémy je analogický ako v predchádzajúcej časti.



\subsection{Zariadenie LMOT03b, schéma \texttt{LMOT03b\_start.slx}}

Zariadenie \textsf{LMOT03b} je pripojené k počítačovej zostave \textsf{LK22}. Zodpovedajúca Simulink schéma je \texttt{LMOT03b\_start.slx}. Opis schémy je analogický ako v predchádzajúcej časti.


\subsection{Zariadenie LMOT01a, schéma \texttt{LMOT01a\_start.slx}}

Zariadenie \textsf{LMOT01a} je pripojené k počítačovej zostave \textsf{LK10}. Zodpovedajúca Simulink schéma je \texttt{LMOT01a\_start.slx}. Opis schémy je analogický ako v predchádzajúcej časti.











































% -----------------------------------------------------------------------------

\end{document}

% -----------------------------------------------------------------------------