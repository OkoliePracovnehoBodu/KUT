\documentclass[a4paper, 10pt, ]{article}

\usepackage[slovak]{babel}

% ------------------------------

\usepackage[utf8]{inputenc}
\usepackage[T1]{fontenc}

\usepackage[left=4cm,
			right=4cm,
			top=2.1cm,
			bottom=2.6cm,
			footskip=7.5mm,
			twoside,
			marginparwidth=3.0cm,
			%showframe,
			]{geometry}

\usepackage{graphicx}
\usepackage[dvipsnames]{xcolor}
% https://en.wikibooks.org/wiki/LaTeX/Colors

% ------------------------------

\usepackage{lmodern}

\usepackage[tt={oldstyle=false,proportional=true,monowidth}]{cfr-lm}
% https://mirror.szerverem.hu/ctan/fonts/cfr-lm/doc/cfr-lm.pdf

% ------------------------------

\usepackage{amsmath}
\usepackage{amssymb}
\usepackage{amsthm}

\usepackage{booktabs}
\usepackage{multirow}
\usepackage{array}
\usepackage{dcolumn}

% \usepackage{natbib}





% ------------------------------

\hyphenpenalty=6000
\tolerance=1000

\def\naT{\mathsf{T}}

% ------------------------------

\makeatletter

	\def\@seccntformat#1{\protect\makebox[0pt][r]{\csname the#1\endcsname\hspace{4mm}}}

	\def\cleardoublepage{\clearpage\if@twoside \ifodd\c@page\else
	\hbox{}
	\vspace*{\fill}
	\begin{center}
	\phantom{}
	\end{center}
	\vspace{\fill}
	\thispagestyle{empty}
	\newpage
	\if@twocolumn\hbox{}\newpage\fi\fi\fi}

	\newcommand\figcaption{\def\@captype{figure}\caption}
	\newcommand\tabcaption{\def\@captype{table}\caption}

\makeatother

% ------------------------------

\usepackage{fancyhdr}
\fancypagestyle{plain}{%
\fancyhf{} % clear all header and footer fields
% \fancyfoot[C]{\sffamily {\bfseries \thepage}\ | {\scriptsize\oznacenieCasti}}
\fancyfoot[C]{\sffamily {\bfseries \thepage}{\color{Gray}\scriptsize$\,$z$\,$\pageref{LastPage}}\ | \includegraphics[height=5pt]{../../KUT000/KUT_logo_v0.1.pdf}{\scriptsize\KUTporadoveCislo}}
\renewcommand{\headrulewidth}{0pt}
\renewcommand{\footrulewidth}{0pt}}
\pagestyle{plain}

% ------------------------------

\usepackage{titlesec}
\titleformat{\paragraph}[hang]{\sffamily  \bfseries}{}{0pt}{}
\titlespacing*{\paragraph}{0mm}{3mm}{1mm}
\titlespacing*{\subparagraph}{0mm}{3mm}{1mm}

\titleformat*{\section}{\sffamily\Large\bfseries}
\titleformat*{\subsection}{\sffamily\large\bfseries}
\titleformat*{\subsubsection}{\sffamily\normalsize\bfseries}


% ------------------------------

\PassOptionsToPackage{hyphens}{url}
\usepackage[pdfauthor={},
			pdftitle={},
			pdfsubject={},
			pdfkeywords={},
			% hidelinks,
			colorlinks=false,
			breaklinks,
			]{hyperref}


% ------------------------------

\graphicspath{%
{./COMMONFILES/}%
{../SVG/}%
{../PY/fig/}%
{../PY/jupynotex/fig/}%
{../ML/fig/}%
}

% ------------------------------

\usepackage{enumitem}

\usepackage{lettrine}

% ------------------------------

\usepackage{lastpage}

\usepackage{microtype}

% ------------------------------

% \usepackage[backend=biber,
%             style=numeric,
%             sorting=none,
%             ]{biblatex}
% \DeclareSourcemap{
%     \maps[datatype=bibtex]{
%         \map{
%         \step[fieldset=note, null]
%         }
%         \map{
%         \step[fieldset=file, null]
%         }        
%         % \map{
%         % \step[fieldset=url, null]        
%         % }
%         \map{
%         \step[fieldset=eprint, null]
%         }
%     }
% }
% 
% \addbibresource{./COMMONFILES/biblist.bib}



% ------------------------------



\usepackage{listings}




\renewcommand{\lstlistingname}{Výpis kódu}
\renewcommand{\lstlistlistingname}{Výpisy kódu}


%New colors defined below
\definecolor{codegreen}{rgb}{0,0.6,0}
\definecolor{codegray}{rgb}{0.5,0.5,0.5}
\definecolor{codepurple}{rgb}{0.58,0,0.82}
\definecolor{backcolour}{rgb}{0.95,0.95,0.95}

%Code listing style named "mystyle"
\lstdefinestyle{mystyle}{
  backgroundcolor=\color{backcolour},
  commentstyle=\fontfamily{lmtt}\fontsize{8.5pt}{8.75pt}\selectfont\color{codegreen},
  keywordstyle=\fontfamily{lmtt}\fontsize{8.5pt}{8.75pt}\selectfont\bfseries\color{Blue},
  stringstyle=\fontfamily{lmtt}\fontsize{8.5pt}{8.75pt}\selectfont\color{codepurple},
  basicstyle=\fontfamily{lmtt}\fontsize{8.5pt}{8.75pt}\selectfont,
  breakatwhitespace=false,
  breaklines=true,
  captionpos=t,
  keepspaces=true,
  numbers=left,
  numbersep=4mm,
  numberstyle=\fontfamily{lmtt}\fontsize{8.5pt}{8.75pt}\selectfont\color{lightgray},
  showspaces=false,
  showstringspaces=false,
  showtabs=false,
  tabsize=2,
  % xleftmargin=10pt,
  framesep=10pt,
  language=Python,
  escapechar=|,
}


\lstset{
    inputencoding=utf8,
    extendedchars=true,
    literate=%
    {á}{{\'a}}1
    {č}{{\v{c}}}1
    {ď}{{\v{d}}}1
    {é}{{\'e}}1
    {ě}{{\v{e}}}1
    {í}{{\'i}}1
    {ň}{{\v{n}}}1
    {ó}{{\'o}}1
    {ř}{{\v{r}}}1
    {š}{{\v{s}}}1
    {ť}{{\v{t}}}1
    {ú}{{\'u}}1
    {ů}{{\r{u}}}1
    {ý}{{\'y}}1
    {ž}{{\v{z}}}1
    {Á}{{\'A}}1
    {Č}{{\v{C}}}1
    {Ď}{{\v{D}}}1
    {É}{{\'E}}1
    {Ě}{{\v{E}}}1
    {Í}{{\'I}}1
    {Ň}{{\v{N}}}1
    {Ó}{{\'O}}1
    {Ř}{{\v{R}}}1
    {Š}{{\v{S}}}1
    {Ť}{{\v{T}}}1
    {Ú}{{\'U}}1
    {Ů}{{\r{U}}}1
    {Ý}{{\'Y}}1
    {Ž}{{\v{Z}}}1
    {ô}{{\^{o}}}1
}



\usepackage{caption}

\DeclareCaptionFormat{odsadene}{\protect\makebox[0pt][r]{#1#2\hspace{4mm}}#3\par}
\DeclareCaptionLabelSeparator{lendvojbodka}{:}
\DeclareCaptionFont{lightgray}{\fontfamily{lmtt}\fontsize{8.5pt}{8.75pt}\selectfont\color{lightgray}}

\captionsetup[lstlisting]{format=odsadene, labelsep=lendvojbodka, justification=raggedright, singlelinecheck=false, labelfont={sf, lightgray},}


% ------------------------------

\usepackage{dirtree}


% ------------------------------



% -----------------------------------------------------------------------------

\def\oznacenieCelku{Kolekcia učebných textov}

% -----------------------------------------------------------------------------


\def\KUTporadoveCislo{018}

\def\oznacenieVerzie{v0.96}
% \def\oznacenieVerzie{\phantom{v1.0}}

\def\mesiacRok{jún 2025}

\def\authorslabel{MT}






% -----------------------------------------------------------------------------

\begin{document}

% -----------------------------------------------------------------------------
% Uvodny nadpis

\noindent
\parbox[t][18mm][c]{0.3\textwidth}{%
\raisebox{-0.9\height}{%
\phantom{.}\includegraphics[height=18mm]{../../KUT000/URKFEIlogo_v0.1.pdf}%
}%
}%
\parbox[t][18mm][c]{0.7\textwidth}{%
\raggedleft

\sffamily
\fontsize{16pt}{18pt}
\fontseries{sbc}
\selectfont

\noindent
\textcolor[rgb]{0.75, 0.75, 0.75}{\textls[25]{\oznacenieCelku}}
}%

\noindent
\parbox[t][16mm][b]{0.5\textwidth}{%
\raggedright

\color{Gray}
\sffamily

\fontsize{12pt}{12pt}
\selectfont
\mesiacRok

\fontsize{6pt}{10pt}
\selectfont
github.com/OkoliePracovnehoBodu/KUT

\fontsize{8pt}{10pt}
\selectfont
\authorslabel




}%
\parbox[t][16mm][b]{0.5\textwidth}{%
\raggedleft

\sffamily

\fontsize{6pt}{6pt}
\selectfont

\textcolor[rgb]{0.68, 0.68, 0.68}{\oznacenieVerzie}


\fontsize{14pt}{14pt}
\selectfont

\bfseries

\includegraphics[height=12pt]{../../KUT000/KUT_logo_v0.1.pdf}%
{%
\textls[-50]{\KUTporadoveCislo}
}%
}%

% -----------------------------------------------------------------------------




\vspace{6mm}

% ---------------------------------------------
\sffamily
\bfseries
\fontsize{18pt}{21pt}
\selectfont

\begin{flushleft}
    LMOT: Príklad datasetu pre prevodovú charakteristiku
\end{flushleft}

\bigskip

% -----------------------------------------------------------------------------
\normalsize
\normalfont
% -----------------------------------------------------------------------------

\lstset{style=mystyle}










\noindent
\lettrine[lines=1, nindent=1pt, loversize=0.0]{C}{ieľom} 
textu je sprostredkovanie a opis meraní a dát súvisiacich so statickými vlastnosťami laboratórneho dynamického systému LMOT. Predchádzajúce texty súvisiace so zariadením LMOT: \href{run:../../KUT014/TeX/KUT014.pdf}{\textsf{KUT014}}, \href{run:../../KUT016/TeX/KUT016.pdf}{\textsf{KUT016}} 







\section{Meranie prevodovej charakteristiky}

V kontexte statických vlastností systému má vo všeobecnosti význam hovoriť o~prevodovej charakteristike systému. Prevodová charakteristika je závislosť ustálených hodnôt výstupného signálu systému od ustálených hodnôt vstupného signálu systému.

Je zrejmé, že prevodová charakteristika sa týka systémov s prívlastkom statické, teda takých, ktoré nie sú astatické.

Prevodová charakteristika, niekde sa nazýva aj statická charakteristika, teda charakterizuje systém len v~ustálených stavoch. Neobsahuje informáciu o dynamike systému.



\subsection{Návrh merania}

Z opisu predmetného dynamického systému [\href{run:../../KUT014/TeX/KUT014.pdf}{\textsf{KUT014}}] vyplýva, že systém má jeden výstupný signál, jeden vstupný signál a manuálne nastaviteľnú prevádzkovú podmienku.

Vstupný a výstupný signál nadobúdajú hodnoty v~rozsahu $0$ až $10$ pričom ide o~napäťové signály vo voltoch [V].

Prevádzková podmienka systému sa nastavuje manuálne otáčaním potenciometra. Signál o polohe potenciometra nadobúda hodnoty v rozsahu $0$ [V] až $10$ [V].





\subsubsection{Voľba ustálených hodnôt vstupov}

O predmetnom systéme je známe, že výstup systému sa ustáli vždy ak sú vstupy systému ustálené. Pre vyšetrovanie ustálených stavov je teda možné využiť celý rozsah vstupného signálu a celý rozsah prevádzkových podmienok.

Návrh uvažuje ustálené hodnoty vstupného signálu uvedené v tabuľke~\ref{tab:ustalene_hodnoty_vstupneho_signalu} a zároveň ustálené hodnoty reprezentujúce prevádzkové podmienky podľa tabuľky \ref{tab:ustalene_hodnoty_podmienok}.


\begin{center}

\vspace{-10pt}    
    
\tabcaption{Ustálené hodnoty vstupného signálu [V]}
\label{tab:ustalene_hodnoty_vstupneho_signalu}

\lstyle

\begin{tabular*}{\textwidth}{@{ \extracolsep{\fill}} ccccccccccc}
\toprule
0 & 1 & 2 & 3 & 4 & 5 & 6 & 7 & 8 & 9 & 10 \\
\bottomrule
\end{tabular*}


\tabcaption{Ustálené hodnoty signálu o prevádkových podmienkach [V]}
\label{tab:ustalene_hodnoty_podmienok}

\lstyle

\begin{tabular*}{\textwidth}{@{ \extracolsep{\fill}} ccccccccccc}
\toprule
0 & 1 & 2 & 3 & 4 & 5 & 6 & 7 & 8 & 9 & 10 \\
\bottomrule
\end{tabular*}

\end{center}


\subsubsection{Voľba časového intervalu pre ustálenie výstupu systému}

Empirické skúsenosti so systémom ukazujú, že z praktického hľadiska sa systém ustáli do $15$ sekúnd po zmene na vstupe systému. Ukazuje sa však aj náchylnosť systému k poruchám spôsobeným zväčša mechanickými nedostatkami a vibráciami zrejme spôsobujúcimi zmeny trenia v mechanických častiach systému. Pre pozorovanie a~vyhodnotenie vplyvu týchto porúch v ustálenom stave je časový interval pre ustálenie zvolený na $120$ sekúnd.



\subsubsection{Postup merania}

Vzhľadom na uvedené voľby ustálených hodnôt a časového intervalu návrh predpokladá nasledovný postup.

\begin{enumerate}[leftmargin=0pt, labelsep=4mm, itemsep=0pt]
    \item Manuálne nastavenie prevádzkových podmienok na hodnotu z tabuľky~\ref{tab:ustalene_hodnoty_podmienok}.
    \item Postupná zmena vstupného signálu na hodnoty z tabuľky~\ref{tab:ustalene_hodnoty_vstupneho_signalu} so zvoleným časovým intervalom. Takúto postupnú zmenu vyjadruje nasledujúca tabuľka \ref{tab:postupna_zmena_vstupneho_signalu}.
    
\end{enumerate}


\begin{center}

\vspace{-10pt}    

\tabcaption{Postupná zmena vstupného signálu}
\label{tab:postupna_zmena_vstupneho_signalu}

\lstyle

\begin{tabular*}{\textwidth}{@{ \extracolsep{\fill}} cc}
\toprule
Čas zmeny vstupného signálu [s] & Hodnota vstupného signál [V] \\
\midrule
0 & 0 \\
120 & 1 \\
240 & 2 \\
360 & 3 \\
480 & 4 \\
600 & 5 \\
720 & 6 \\
840 & 7 \\
960 & 8 \\
1080 & 9 \\
1200 & 10 \\
\bottomrule
\end{tabular*}

\end{center}

Celková dĺžka merania je teda $1200 + 120 = 1320$ sekúnd a počas tejto doby sú prevádzkové podmienky konštantné.















\section{Získané dáta}


Na základe uvedeného návrhu a postupu merania boli vytvorené skripty a simulačné schémy v rámci prostredia MATLAB - Simulink. Ich opis a dokumentácia nech sú nad rámec tohto textu. Výsledný dataset sa nachádza v adresári \texttt{ML/dataRepo/dataSet01}.

Vizualizácia týchto dát je realizovaná pomocou Python skriptu v Jupyter notebooku \texttt{PY/job\_makeFigs.ipynb}.

Získané dáta všetky prevádzkové podmienky, ktoré sú v~tabuľke~\ref{tab:ustalene_hodnoty_podmienok}, sú vizualizované na nasledujúcich obrázkoch.















\paragraph{Meranie ustálených hodnôt pri prevádzkovej podmienke 0 [V]}

\begin{center}

    \vbox{%
        \makebox[\textwidth][c]{%
        \includegraphics{fj_01_data_pot0_panel_1.pdf}
        }

        \makebox[\textwidth][c]{%
        \includegraphics{fj_01_data_pot0_panel_3.pdf}
        }

        \makebox[\textwidth][c]{%
        \includegraphics{fj_01_data_pot0_panel_2.pdf}
        }

        \figcaption{ 
           Meranie pri nastavenej prevádzkovej podmienke $0$ [V].
           Dáta na obrázku dostupné ako dátové súbory v~adresári \texttt{PY/fig/fj\_01\_data\_pot0\_panel\_\#.csv}
        }
        \label{fj_01_data_pot0}
    }%vbox

\end{center}

\vfill

\paragraph{Meranie ustálených hodnôt pri prevádzkovej podmienke 1 [V]}

\begin{center}

    \vbox{%
        \makebox[\textwidth][c]{%
        \includegraphics{fj_01_data_pot1_panel_1.pdf}
        }

        \makebox[\textwidth][c]{%
        \includegraphics{fj_01_data_pot1_panel_3.pdf}
        }

        \makebox[\textwidth][c]{%
        \includegraphics{fj_01_data_pot1_panel_2.pdf}
        }

        \figcaption{ 
            Meranie pri nastavenej prevádzkovej podmienke $1$ [V].
            Dáta na obrázku dostupné ako dátové súbory v~adresári \texttt{PY/fig/fj\_01\_data\_pot1\_panel\_\#.csv}
        }

        \label{fj_01_data_pot1}
    }%vbox

\end{center}


\pagebreak














\paragraph{Meranie ustálených hodnôt pri prevádzkovej podmienke 2 [V]}

\begin{center}

    \vbox{%
        \makebox[\textwidth][c]{%
        \includegraphics{fj_01_data_pot2_panel_1.pdf}
        }

        \makebox[\textwidth][c]{%
        \includegraphics{fj_01_data_pot2_panel_3.pdf}
        }

        \makebox[\textwidth][c]{%
        \includegraphics{fj_01_data_pot2_panel_2.pdf}
        }

        \figcaption{ 
           Meranie pri nastavenej prevádzkovej podmienke $2$ [V].
           Dáta na obrázku dostupné ako dátové súbory v~adresári \texttt{PY/fig/fj\_01\_data\_pot2\_panel\_\#.csv}
        }
        \label{fj_01_data_pot2}
    }%vbox

\end{center}

\vfill

\paragraph{Meranie ustálených hodnôt pri prevádzkovej podmienke 3 [V]}

\begin{center}

    \vbox{%
        \makebox[\textwidth][c]{%
        \includegraphics{fj_01_data_pot3_panel_1.pdf}
        }

        \makebox[\textwidth][c]{%
        \includegraphics{fj_01_data_pot3_panel_3.pdf}
        }

        \makebox[\textwidth][c]{%
        \includegraphics{fj_01_data_pot3_panel_2.pdf}
        }

        \figcaption{ 
            Meranie pri nastavenej prevádzkovej podmienke $3$ [V].
            Dáta na obrázku dostupné ako dátové súbory v~adresári \texttt{PY/fig/fj\_01\_data\_pot3\_panel\_\#.csv}
        }

        \label{fj_01_data_pot3}
    }%vbox

\end{center}


\pagebreak











\paragraph{Meranie ustálených hodnôt pri prevádzkovej podmienke 4 [V]}

\begin{center}

    \vbox{%
        \makebox[\textwidth][c]{%
        \includegraphics{fj_01_data_pot4_panel_1.pdf}
        }

        \makebox[\textwidth][c]{%
        \includegraphics{fj_01_data_pot4_panel_3.pdf}
        }

        \makebox[\textwidth][c]{%
        \includegraphics{fj_01_data_pot4_panel_2.pdf}
        }

        \figcaption{ 
           Meranie pri nastavenej prevádzkovej podmienke $4$ [V].
           Dáta na obrázku dostupné ako dátové súbory v~adresári \texttt{PY/fig/fj\_01\_data\_pot4\_panel\_\#.csv}
        }
        \label{fj_01_data_pot4}
    }%vbox

\end{center}

\vfill

\paragraph{Meranie ustálených hodnôt pri prevádzkovej podmienke 5 [V]}

\begin{center}

    \vbox{%
        \makebox[\textwidth][c]{%
        \includegraphics{fj_01_data_pot5_panel_1.pdf}
        }

        \makebox[\textwidth][c]{%
        \includegraphics{fj_01_data_pot5_panel_3.pdf}
        }

        \makebox[\textwidth][c]{%
        \includegraphics{fj_01_data_pot5_panel_2.pdf}
        }

        \figcaption{ 
            Meranie pri nastavenej prevádzkovej podmienke $5$ [V].
            Dáta na obrázku dostupné ako dátové súbory v~adresári \texttt{PY/fig/fj\_01\_data\_pot5\_panel\_\#.csv}
        }

        \label{fj_01_data_pot5}
    }%vbox

\end{center}


\pagebreak














\paragraph{Meranie ustálených hodnôt pri prevádzkovej podmienke 6 [V]}

\begin{center}

    \vbox{%
        \makebox[\textwidth][c]{%
        \includegraphics{fj_01_data_pot6_panel_1.pdf}
        }

        \makebox[\textwidth][c]{%
        \includegraphics{fj_01_data_pot6_panel_3.pdf}
        }

        \makebox[\textwidth][c]{%
        \includegraphics{fj_01_data_pot6_panel_2.pdf}
        }

        \figcaption{ 
           Meranie pri nastavenej prevádzkovej podmienke $6$ [V].
           Dáta na obrázku dostupné ako dátové súbory v~adresári \texttt{PY/fig/fj\_01\_data\_pot6\_panel\_\#.csv}
        }
        \label{fj_01_data_pot6}
    }%vbox

\end{center}

\vfill

\paragraph{Meranie ustálených hodnôt pri prevádzkovej podmienke 7 [V]}

\begin{center}

    \vbox{%
        \makebox[\textwidth][c]{%
        \includegraphics{fj_01_data_pot7_panel_1.pdf}
        }

        \makebox[\textwidth][c]{%
        \includegraphics{fj_01_data_pot7_panel_3.pdf}
        }

        \makebox[\textwidth][c]{%
        \includegraphics{fj_01_data_pot7_panel_2.pdf}
        }

        \figcaption{ 
            Meranie pri nastavenej prevádzkovej podmienke $7$ [V].
            Dáta na obrázku dostupné ako dátové súbory v~adresári \texttt{PY/fig/fj\_01\_data\_pot7\_panel\_\#.csv}
        }

        \label{fj_01_data_pot7}
    }%vbox

\end{center}


\pagebreak









\paragraph{Meranie ustálených hodnôt pri prevádzkovej podmienke 8 [V]}

\begin{center}

    \vbox{%
        \makebox[\textwidth][c]{%
        \includegraphics{fj_01_data_pot8_panel_1.pdf}
        }

        \makebox[\textwidth][c]{%
        \includegraphics{fj_01_data_pot8_panel_3.pdf}
        }

        \makebox[\textwidth][c]{%
        \includegraphics{fj_01_data_pot8_panel_2.pdf}
        }

        \figcaption{ 
           Meranie pri nastavenej prevádzkovej podmienke $8$ [V].
           Dáta na obrázku dostupné ako dátové súbory v~adresári \texttt{PY/fig/fj\_01\_data\_pot8\_panel\_\#.csv}
        }
        \label{fj_01_data_pot8}
    }%vbox

\end{center}

\vfill

\paragraph{Meranie ustálených hodnôt pri prevádzkovej podmienke 9 [V]}

\begin{center}

    \vbox{%
        \makebox[\textwidth][c]{%
        \includegraphics{fj_01_data_pot9_panel_1.pdf}
        }

        \makebox[\textwidth][c]{%
        \includegraphics{fj_01_data_pot9_panel_3.pdf}
        }

        \makebox[\textwidth][c]{%
        \includegraphics{fj_01_data_pot9_panel_2.pdf}
        }

        \figcaption{ 
            Meranie pri nastavenej prevádzkovej podmienke $9$ [V].
            Dáta na obrázku dostupné ako dátové súbory v~adresári \texttt{PY/fig/fj\_01\_data\_pot9\_panel\_\#.csv}
        }

        \label{fj_01_data_pot9}
    }%vbox

\end{center}


\pagebreak







\paragraph{Meranie ustálených hodnôt pri prevádzkovej podmienke 10 [V]}

\begin{center}

    \vbox{%
        \makebox[\textwidth][c]{%
        \includegraphics{fj_01_data_pot10_panel_1.pdf}
        }

        \makebox[\textwidth][c]{%
        \includegraphics{fj_01_data_pot10_panel_3.pdf}
        }

        \makebox[\textwidth][c]{%
        \includegraphics{fj_01_data_pot10_panel_2.pdf}
        }

        \figcaption{ 
           Meranie pri nastavenej prevádzkovej podmienke $10$ [V].
           Dáta na obrázku dostupné ako dátové súbory v~adresári \texttt{PY/fig/fj\_01\_data\_pot10\_panel\_\#.csv}
        }
        \label{fj_01_data_pot10}
    }%vbox

\end{center}









% -----------------------------------------------------------------------------

\end{document}

% -----------------------------------------------------------------------------